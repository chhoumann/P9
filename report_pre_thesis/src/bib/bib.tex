@article{andersonImprovedAccuracyQuantitative2017,
  title = {Improved Accuracy in Quantitative Laser-Induced Breakdown Spectroscopy Using Sub-Models},
  author = {Anderson, Ryan B. and Clegg, Samuel M. and Frydenvang, Jens and Wiens, Roger C. and McLennan, Scott and Morris, Richard V. and Ehlmann, Bethany and Dyar, M. Darby},
  date = {2017-03-01},
  journaltitle = {Spectrochimica Acta Part B: Atomic Spectroscopy},
  shortjournal = {Spectrochimica Acta Part B: Atomic Spectroscopy},
  volume = {129},
  pages = {49--57},
  issn = {0584-8547},
  doi = {10.1016/j.sab.2016.12.002},
  url = {https://www.sciencedirect.com/science/article/pii/S0584854716303925},
  urldate = {2023-02-11},
  abstract = {Accurate quantitative analysis of diverse geologic materials is one of the primary challenges faced by the laser-induced breakdown spectroscopy (LIBS)-based ChemCam instrument on the Mars Science Laboratory (MSL) rover. The SuperCam instrument on the Mars 2020 rover, as well as other LIBS instruments developed for geochemical analysis on Earth or other planets, will face the same challenge. Consequently, part of the ChemCam science team has focused on the development of improved multivariate analysis calibrations methods. Developing a single regression model capable of accurately determining the composition of very different target materials is difficult because the response of an element's emission lines in LIBS spectra can vary with the concentration of other elements. We demonstrate a conceptually simple “sub-model” method for improving the accuracy of quantitative LIBS analysis of diverse target materials. The method is based on training several regression models on sets of targets with limited composition ranges and then “blending” these “sub-models” into a single final result. Tests of the sub-model method show improvement in test set root mean squared error of prediction (RMSEP) for almost all cases. The sub-model method, using partial least squares (PLS) regression, is being used as part of the current ChemCam quantitative calibration, but the sub-model method is applicable to any multivariate regression method and may yield similar improvements.},
  langid = {english},
}

@article{cleggRecalibrationMarsScience2017,
  title = {Recalibration of the {{Mars Science Laboratory ChemCam}} Instrument with an Expanded Geochemical Database},
  author = {Clegg, Samuel M. and Wiens, Roger C. and Anderson, Ryan and Forni, Olivier and Frydenvang, Jens and Lasue, Jeremie and Cousin, Agnes and Payré, Valérie and Boucher, Tommy and Dyar, M. Darby and McLennan, Scott M. and Morris, Richard V. and Graff, Trevor G. and Mertzman, Stanley A. and Ehlmann, Bethany L. and Belgacem, Ines and Newsom, Horton and Clark, Ben C. and Melikechi, Noureddine and Mezzacappa, Alissa and McInroy, Rhonda E. and Martinez, Ronald and Gasda, Patrick and Gasnault, Olivier and Maurice, Sylvestre},
  date = {2017-03-01},
  journaltitle = {Spectrochimica Acta Part B: Atomic Spectroscopy},
  shortjournal = {Spectrochimica Acta Part B: Atomic Spectroscopy},
  volume = {129},
  pages = {64--85},
  issn = {0584-8547},
  doi = {10.1016/j.sab.2016.12.003},
  url = {https://www.sciencedirect.com/science/article/pii/S0584854716303913},
  urldate = {2023-02-11},
  abstract = {The ChemCam Laser-Induced Breakdown Spectroscopy (LIBS) instrument onboard the Mars Science Laboratory (MSL) rover Curiosity has obtained {$>$}300,000 spectra of rock and soil analysis targets since landing at Gale Crater in 2012, and the spectra represent perhaps the largest publicly-available LIBS datasets. The compositions of the major elements, reported as oxides (SiO2, TiO2, Al2O3, FeOT, MgO, CaO, Na2O, K2O), have been re-calibrated using a laboratory LIBS instrument, Mars-like atmospheric conditions, and a much larger set of standards (408) that span a wider compositional range than previously employed. The new calibration uses a combination of partial least squares (PLS1) and Independent Component Analysis (ICA) algorithms, together with a calibration transfer matrix to minimize differences between the conditions under which the standards were analyzed in the laboratory and the conditions on Mars. While the previous model provided good results in the compositional range near the average Mars surface composition, the new model fits the extreme compositions far better. Examples are given for plagioclase feldspars, where silicon was significantly over-estimated by the previous model, and for calcium-sulfate veins, where silicon compositions near zero were inaccurate. The uncertainties of major element abundances are described as a function of the abundances, and are overall significantly lower than the previous model, enabling important new geochemical interpretations of the data.},
  langid = {english},
}

@article{wiens_pre-flight_2013,
	title = {Pre-flight calibration and initial data processing for the {ChemCam} laser-induced breakdown spectroscopy instrument on the Mars Science Laboratory rover},
	volume = {82},
	issn = {0584-8547},
	url = {https://www.sciencedirect.com/science/article/pii/S0584854713000505},
	doi = {10.1016/j.sab.2013.02.003},
	abstract = {The {ChemCam} instrument package on the Mars Science Laboratory rover, Curiosity, is the first planetary science instrument to employ laser-induced breakdown spectroscopy ({LIBS}) to determine the compositions of geological samples on another planet. Pre-processing of the spectra involves subtracting the ambient light background, removing noise, removing the electron continuum, calibrating for the wavelength, correcting for the variable distance to the target, and applying a wavelength-dependent correction for the instrument response. Further processing of the data uses multivariate and univariate comparisons with a {LIBS} spectral library developed prior to launch as well as comparisons with several on-board standards post-landing. The level-2 data products include semi-quantitative abundances derived from partial least squares regression. A {LIBS} spectral library was developed using 69 rock standards in the form of pressed powder disks, glasses, and ceramics to minimize heterogeneity on the scale of the observation (350–550μm dia.). The standards covered typical compositional ranges of igneous materials and also included sulfates, carbonates, and phyllosilicates. The provenance and elemental and mineralogical compositions of these standards are described. Spectral characteristics of this data set are presented, including the size distribution and integrated irradiances of the plasmas, and a proxy for plasma temperature as a function of distance from the instrument. Two laboratory-based clones of {ChemCam} reside in Los Alamos and Toulouse for the purpose of adding new spectra to the database as the need arises. Sensitivity to differences in wavelength correlation to spectral channels and spectral resolution has been investigated, indicating that spectral registration needs to be within half a pixel and resolution needs to match within 1.5 to 2.6pixels. Absolute errors are tabulated for derived compositions of each major element in each standard using {PLS} regression. Sources of errors are investigated and discussed, and methods for improving the analytical accuracy of compositions derived from {ChemCam} spectra are discussed.},
	pages = {1--27},
	journaltitle = {Spectrochimica Acta Part B: Atomic Spectroscopy},
	shortjournal = {Spectrochimica Acta Part B: Atomic Spectroscopy},
	author = {Wiens, R. C. and Maurice, S. and Lasue, J. and Forni, O. and Anderson, R. B. and Clegg, S. and Bender, S. and Blaney, D. and Barraclough, B. L. and Cousin, A. and Deflores, L. and Delapp, D. and Dyar, M. D. and Fabre, C. and Gasnault, O. and Lanza, N. and Mazoyer, J. and Melikechi, N. and Meslin, P. -Y. and Newsom, H. and Ollila, A. and Perez, R. and Tokar, R. L. and Vaniman, D.},
	urldate = {2023-10-02},
	date = {2013-04-01},
	keywords = {Laser-induced breakdown spectroscopy, {ChemCam}, Curiosity rover, {LIBS}, Mars},
	file = {ScienceDirect Full Text PDF:C\:\\Users\\Patrick\\Zotero\\storage\\Q2FRVJXC\\Wiens et al. - 2013 - Pre-flight calibration and initial data processing.pdf:application/pdf;ScienceDirect Snapshot:C\:\\Users\\Patrick\\Zotero\\storage\\ER8DPMQ6\\S0584854713000505.html:text/html},
}

@misc{chemcam_nasa_website, title={ChemCam}, url={https://mars.nasa.gov/msl/spacecraft/instruments/chemcam/}, journal={NASA}, publisher={NASA}, author={Lanza, Nina}, year={2022}, month={May}}

@misc{curiosity_nasa_website, title={Mars Curiosity Rover}, url={https://mars.nasa.gov/msl/home/}, journal={NASA}, publisher={NASA}, author={NASA}, year={2021}, month={Sep}}

@article{wiens_chemcam_2012,
	title = {The {ChemCam} Instrument Suite on the Mars Science Laboratory ({MSL}) Rover: Body Unit and Combined System Tests},
	volume = {170},
	issn = {1572-9672},
	url = {https://doi.org/10.1007/s11214-012-9902-4},
	doi = {10.1007/s11214-012-9902-4},
	shorttitle = {The {ChemCam} Instrument Suite on the Mars Science Laboratory ({MSL}) Rover},
	abstract = {The {ChemCam} instrument suite on the Mars Science Laboratory ({MSL}) rover Curiosity provides remote compositional information using the first laser-induced breakdown spectrometer ({LIBS}) on a planetary mission, and provides sample texture and morphology data using a remote micro-imager ({RMI}). Overall, {ChemCam} supports {MSL} with five capabilities: remote classification of rock and soil characteristics; quantitative elemental compositions including light elements like hydrogen and some elements to which {LIBS} is uniquely sensitive (e.g., Li, Be, Rb, Sr, Ba); remote removal of surface dust and depth profiling through surface coatings; context imaging; and passive spectroscopy over the 240–905 nm range. {ChemCam} is built in two sections: The mast unit, consisting of a laser, telescope, {RMI}, and associated electronics, resides on the rover’s mast, and is described in a companion paper. {ChemCam}’s body unit, which is mounted in the body of the rover, comprises an optical demultiplexer, three spectrometers, detectors, their coolers, and associated electronics and data handling logic. Additional instrument components include a 6 m optical fiber which transfers the {LIBS} light from the telescope to the body unit, and a set of onboard calibration targets. {ChemCam} was integrated and tested at Los Alamos National Laboratory where it also underwent {LIBS} calibration with 69 geological standards prior to integration with the rover. Post-integration testing used coordinated mast and instrument commands, including {LIBS} line scans on rock targets during system-level thermal-vacuum tests. In this paper we describe the body unit, optical fiber, and calibration targets, and the assembly, testing, and verification of the instrument prior to launch.},
	pages = {167--227},
	number = {1},
	journaltitle = {Space Science Reviews},
	shortjournal = {Space Sci Rev},
	author = {Wiens, Roger C. and Maurice, Sylvestre and Barraclough, Bruce and Saccoccio, Muriel and Barkley, Walter C. and Bell, James F. and Bender, Steve and Bernardin, John and Blaney, Diana and Blank, Jennifer and Bouyé, Marc and Bridges, Nathan and Bultman, Nathan and Caïs, Phillippe and Clanton, Robert C. and Clark, Benton and Clegg, Samuel and Cousin, Agnes and Cremers, David and Cros, Alain and {DeFlores}, Lauren and Delapp, Dorothea and Dingler, Robert and D’Uston, Claude and Darby Dyar, M. and Elliott, Tom and Enemark, Don and Fabre, Cecile and Flores, Mike and Forni, Olivier and Gasnault, Olivier and Hale, Thomas and Hays, Charles and Herkenhoff, Ken and Kan, Ed and Kirkland, Laurel and Kouach, Driss and Landis, David and Langevin, Yves and Lanza, Nina and {LaRocca}, Frank and Lasue, Jeremie and Latino, Joseph and Limonadi, Daniel and Lindensmith, Chris and Little, Cynthia and Mangold, Nicolas and Manhes, Gerard and Mauchien, Patrick and {McKay}, Christopher and Miller, Ed and Mooney, Joe and Morris, Richard V. and Morrison, Leland and Nelson, Tony and Newsom, Horton and Ollila, Ann and Ott, Melanie and Pares, Laurent and Perez, René and Poitrasson, Franck and Provost, Cheryl and Reiter, Joseph W. and Roberts, Tom and Romero, Frank and Sautter, Violaine and Salazar, Steven and Simmonds, John J. and Stiglich, Ralph and Storms, Steven and Striebig, Nicolas and Thocaven, Jean-Jacques and Trujillo, Tanner and Ulibarri, Mike and Vaniman, David and Warner, Noah and Waterbury, Rob and Whitaker, Robert and Witt, James and Wong-Swanson, Belinda},
	urldate = {2023-10-03},
	date = {2012-09-01},
	langid = {english},
	keywords = {{ChemCam}, Curiosity, Gale Crater, Laser induced breakdown spectroscopy, Laser plasma, {LIBS}, Mars, Mars Science Laboratory, {MSL}, {RMI}},
	file = {Full Text PDF:C\:\\Users\\Patrick\\Zotero\\storage\\SGFN2FRN\\Wiens et al. - 2012 - The ChemCam Instrument Suite on the Mars Science L.pdf:application/pdf},
}

@article{knight_characterization_2000,
	title = {Characterization of Laser-Induced Breakdown Spectroscopy ({LIBS}) for Application to Space Exploration},
	volume = {54},
	issn = {0003-7028, 1943-3530},
	url = {http://journals.sagepub.com/doi/10.1366/0003702001949591},
	doi = {10.1366/0003702001949591},
	abstract = {Early in the next century, several space missions are planned with the goal of landing craft on asteroids, comets, the Moon, and Mars. To increase the scientific return of these missions, new methods are needed to provide (1) significantly more analyses per mission lifetime, and (2) expanded analytical capabilities. One method that has the potential to meet both of these needs for the elemental analysis of geological samples is laser-induced breakdown spectroscopy ({LIBS}). These capabilities are possible because the laser plasma provides rapid analysis and the laser pulse can be focused on a remotely located sample to perform a stand-off measurement. Stand-off is defined as a distance up to 20 m between the target and laser. Here we present the results of a characterization of {LIBS} for the stand-off analysis of soils at reduced air pressures and in a simulated Martian atmosphere (5–7 torr pressure of {CO}
              2
              ) showing the feasibility of {LIBS} for space exploration. For example, it is demonstrated that an analytically useful laser plasma can be generated at distances up to 19 m by using only 35 {mJ}/pulse from a compact laser. Some characteristics of the laser plasma at reduced pressure were also investigated. Temporally and spectrally resolved imaging showed significant changes in the plasma as the pressure was reduced and also showed that the analyte signals and mass ablated from a target were strongly dependent on pressure. As the pressure decreased from 590 torr to the 40–100 torr range, the signals increased by a factor of about 3–4, and as the pressure was further reduced the signals decreased. This behavior can be explained by pressure-dependent changes in the mass of material vaporized and the frequency of collisions between species in the plasma. Changes in the temperature and the electron density of the plasmas with pressure were also examined and detection limits for selected elements were determined.},
	pages = {331--340},
	number = {3},
	journaltitle = {Applied Spectroscopy},
	shortjournal = {Appl Spectrosc},
	author = {Knight, Andrew K. and Scherbarth, Nancy L. and Cremers, David A. and Ferris, Monty J.},
	urldate = {2023-10-03},
	date = {2000-03},
	langid = {english},
	file = {Knight et al. - 2000 - Characterization of Laser-Induced Breakdown Spectr.pdf:C\:\\Users\\Patrick\\Zotero\\storage\\CWWBB9JC\\Knight et al. - 2000 - Characterization of Laser-Induced Breakdown Spectr.pdf:application/pdf},
}

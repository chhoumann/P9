\begin{abstract}
     Brief summary of the objectives, methodology, main findings, and significance of the report. The whole story - in short.
\end{abstract}

% Acknowledgments
% - Acknowledge any contributions, support, or guidance from individuals, organizations, or institutions.

% TODO: Convert to IEEE format

\section{Introduction}
Brief introduction to the field of study.
Importance of establishing benchmarks.
Objectives of the first part of the project.
The whole story – longer version but still short, also motivates.

\subsection{Problem Statement Proposal}
\textit{This section is a proposal for the problem statement for the P9 project. It is not a part of the final report.}

\vspace{0.5cm}
\noindent
While current calibration methods for ChemCam have proven effective, there may still be areas of optimization or enhancement.
Discrepancies in results and potential application limitations highlight the need for a critical reassessment.
A comprehensive study is required to recreate and evaluate state-of-the-art methodologies to confirm their reported efficacy.
Understanding the intricacies of spectroscopy data is pivotal for advancements in this domain.
Our objective is to investigate the reproducibility of existing calibration approaches, understand their limitations, and set the foundation for future enhancements.

This leads to the following problem statement:

\begin{quote}
    Are there potential limitations in the current ChemCam calibration approach that could be addressed with state-of-the-art methods?
\end{quote}

\section{Background}
% Background / Preliminaries (what you need to know in order to understand the story)

% What is the purpose of the MSL mission? What is the purpose of the ChemCam instrument?

The NASA ChemCam (Chemistry and Camera) is an advanced instrument developed by NASA in collaboration with the French national space agency (CNES).
It is a remote-sensing laser instrument designed to analyze the chemical composition of rocks and soil on Mars\cite{chemcam_nasa_website}.
The instrument is mounted on the Curiosity rover and is used to collect data in the Gale Crater on Mars\cite{curiosity_nasa_website}.
It leverages the ChemCam-LIBS model, which uses machine learning models to analyze the spectral data obtained from ChemCam.\cite{andersonImprovedAccuracyQuantitative2017}\cite{cleggRecalibrationMarsScience2017}

The use of LIBS (Laser-Induced Breakdown Spectroscopy) technology in planetary exploration has proven to be effective in analyzing soil and rock samples \cite{knight_characterization_2000}.

Initially, a laser is fired multiple times to ablate and remove any surface contaminants, such as dust and weathering layers, to expose the underlying material.
Afterwards, the laser generates a plasma plume from the now-exposed sample material.
This plasma plume emits light, which, when collected and analyzed, reveals the elemental composition of the sample by correlating the intensity of emitted light with specific wavelengths in a LIBS spectrum.
The LIBS technique enables remote analyses of materials without the need for sample preparation.
It allows for rapid analyses because of the immediate spectrum collection from the subsequent plasma, while maintaining a high spatial resolution due to its small observation footprints.
This high resolution is essential for pinpointing and investigating small features. \cite{wiens_chemcam_2012}

In 2013, \citeauthor{wiens_pre-flight_2013} published a paper describing the pre-flight calibration and initial data processing for the ChemCam LIBS instrument.
This paper introduces methods for preprocessing spectra samples and a regression model based on Partial Least Squares (PLS2) used to predict the composition of geological samples on Mars.
The model was trained on a dataset of 69 rock samples from Earth, which were created in a laboratory to simulate the conditions of the Martian surface.
This dataset is often referred to as the calibration dataset.

Two key conclusions were drawn from this paper:
\begin{enumerate}
    \item A larger dataset is needed to improve the accuracy of the model.
    \item The PLS2 model is not ideal for this type of data, and an argument is made for using PLS1 instead.
\end{enumerate}

Based on this work, \citeauthor{cleggRecalibrationMarsScience2017} published a paper in 2017 describing a new approach to the ChemCam LIBS calibration model.
This paper introduces a new model based on PLS1 and Independent Component Analysis (ICA).
In addition, a much larger calibration dataset was used, consisting of 408 samples.
Using this, the team was able to improve the predictions by employing a \textit{submodel} PLS approach in tandem with ICA.
This model is referred to as the MOC (Multivariate Oxide Composition) model.
The MOC model is currently used by the ChemCam team to analyze the LIBS data collected by the Curiosity rover.

\subsection{Domain Background}
Describes the non-CS part of the domain/terminology that is necessary to understand the problem.
This includes relevant physics and ChemCam terminology.

Detailed explanation of the LIBS setup, including equipment, configurations, and settings.
Explain any variables, controls, and calibrations involved in the setup.

\subsection{CS Background for ChemCam calibration}
Current use of CS and statistical methods including PLS1 and ICA.
Description of how these are used in the current model.

\section{State of the Art}
Current leading approaches and methods in chemometric data analysis and prediction.
Brief description of the key technologies or techniques.

A review and analysis of the current model used by NASA, as reported in a particular paper.
Briefly address its capabilities and limitations.

Related Work (What others have done and why our method is different / novel)

\section{Methodology}
Method / Our Contribution (Detailed main part of the story, or how the analysis was done)

\subsection{Data Analysis}
Description of the samples used and their relevance.
Explain how and why these samples were chosen.

\section{Results}
Analysis / Experimental / Empirical Evaluation (the analysis, comparing our method to SoA or a baseline, ablation study)
\begin{itemize}
    \item System used (for repeatability)
    \item Datasets / people used
    \item Metrics used
    \item Results
    \item Discussion
\end{itemize}

\section{Conclusion}
Conclusion (the story again, in short, emphasizing the results)

Summary of the main findings of the report.
Reiteration of the significance of the established benchmarks to the subsequent part of the project.

\section{Recommendations for Future Work}
Suggestions for potential improvements or modifications in the methodology.
Identification of any additional benchmarks that may be relevant for future studies.

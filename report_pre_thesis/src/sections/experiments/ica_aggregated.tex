\subsection{Experiment: ICA Aggregated Datasets}\label{sec:experiment_ica_aggregated_datasets}
In Section~\ref{sec:ica_data_preprocessing}, we described how we only use one of the five location datasets for each sample for the ICA training process.
In this experiment, we used all five location datasets for each sample, and aggregated the results by taking the mean of the shots over the datasets.
Since this results in an averaged dataset, we expect to either lose information by averaging out the differences between the datasets, or gain information by reducing the noise in the dataset.

Table \ref{tab:ica_aggregated_rmses} shows the RMSEs for the ICA model with and without aggregated datasets.

\begin{table}[h]
\centering
\begin{tabular}{lll}
\hline
Element    & ICA baseline   & Aggregated datasets \\
\hline
\ce{SiO2}  & \textbf{10.68} & 12.01 \\
\ce{TiO2}  & 0.63           & \textbf{0.60} \\
\ce{Al2O3} & 5.55           & \textbf{4.81} \\
\ce{FeO_T} & \textbf{8.30}  & 8.56 \\
\ce{MgO}   & 2.90           & \textbf{2.51} \\
\ce{CaO}   & \textbf{3.52}  & 3.71 \\
\ce{Na2O}  & 1.72           & \textbf{1.41} \\
\ce{K2O}   & \textbf{1.37}  & 1.51 \\
\hline
\end{tabular}
\caption{RMSEs for the ICA phase's regression models using aggregated datasets.}
\label{tab:ica_aggregated_rmses}
\end{table}

The results, as depicted in Table~\ref{tab:ica_aggregated_rmses}, indicate little difference in the Root Mean Square Error (RMSE) across all elements when aggregated datasets were used compared to the baseline.
For half of the oxides (\ce{TiO2}, \ce{Al2O3}, \ce{MgO}, \ce{Na2O}), the RMSE is lower when using aggregated datasets, while for the other half (\ce{SiO2}, \ce{FeO_T}, \ce{CaO}, \ce{K2O}), the RMSE is higher.
This suggests that in some cases, the aggregation process may have reduced noise, while in other cases, it may have led to a loss of information necessary for accurate predictions.
In addition, because we do not perform outlier detection for ICA, the aggregated datasets may have been more susceptible to outliers across locations, which could increase the RMSE for samples with high variability.
As evident from the table, the oxides mentioned in \ref{sec:data_overview} whose concentrations are highly variable across samples (like \ce{SiO2} and \ce{FeO_T}) have higher RMSEs when using aggregated datasets, which supports this notion.
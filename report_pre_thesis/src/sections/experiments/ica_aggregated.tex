\subsection{Experiment: ICA Aggregated Datasets}\label{sec:experiment_ica_aggregated_datasets}
In Section~\ref{sec:ica_data_preprocessing}, we described how we only use one of the five location datasets for each sample for the ICA training process.
In this experiment, we use all five location datasets for each sample, and aggregate the results by taking the mean of the shots over the datasets.
Since this results in an averaged dataset, we expect to either lose information by averaging out the differences between the datasets, or gain information by reducing the noise in the dataset.

Table \ref{tab:ica_aggregated_rmses} shows the RMSEs for the ICA model with and without aggregated datasets.

\begin{table}[h]
\centering
\begin{tabular}{lll}
\hline
Element    & Baseline      & Aggregated datasets \\
\hline
\ce{SiO2}  & 5.81          & ? \\
\ce{TiO2}  & 0.47          & ? \\
\ce{Al2O3} & 1.94          & ? \\
\ce{FeO_T} & 4.35          & ? \\
\ce{MgO}   & 1.17          & ? \\
\ce{CaO}   & 1.43          & ? \\
\ce{Na2O}  & 0.66          & ? \\
\ce{K2O}   & 0.72          & ? \\
\hline
\end{tabular}
\caption{RMSEs for the ICA phase's regression models using aggregated datasets.}
\label{tab:ica_aggregated_rmses}
\end{table}
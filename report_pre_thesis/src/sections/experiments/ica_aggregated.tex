\subsection{Experiment: ICA Aggregated Datasets}\label{sec:experiment_ica_aggregated_datasets}
In Section~\ref{sec:ica_data_preprocessing}, we described how we only use one of the five location datasets for each sample for the ICA training process.
In this experiment, we used all five location datasets for each sample, and aggregated the results by taking the mean of the shots over the datasets.
Since this results in an averaged dataset, we expect to either lose information by averaging out the differences between the datasets, or gain information by reducing the noise in the dataset.

Table \ref{tab:ica_aggregated_rmses} shows the RMSEs for the ICA model with and without aggregated datasets.

\begin{table}[h]
\centering
\begin{tabular}{lll}
\hline
Element    & ICA baseline   & Aggregated datasets \\
\hline
\ce{SiO2}  & \textbf{10.68} & 12.01 \\
\ce{TiO2}  & 0.63           & \textbf{0.60} \\
\ce{Al2O3} & 5.55           & \textbf{4.81} \\
\ce{FeO_T} & \textbf{8.30}  & 8.56 \\
\ce{MgO}   & 2.90           & \textbf{2.51} \\
\ce{CaO}   & \textbf{3.52}  & 3.71 \\
\ce{Na2O}  & 1.72           & \textbf{1.41} \\
\ce{K2O}   & \textbf{1.37}  & 1.51 \\
\hline
\end{tabular}
\caption{RMSEs for the ICA phase's regression models using aggregated datasets.}
\label{tab:ica_aggregated_rmses}
\end{table}

The results, as depicted in Table~\ref{tab:ica_aggregated_rmses}, indicate a significant increase in the Root Mean Square Error (RMSE) across all elements when aggregated datasets were used compared to the baseline.
Specifically, the RMSE values for \ce{SiO2}, \ce{Al2O3}, \ce{FeO_T}, \ce{MgO}, \ce{CaO}, \ce{Na2O}, and \ce{K2O} all saw substantial increases.
This suggests that the aggregation process, contrary to potentially enhancing model accuracy by noise reduction, actually led to a considerable loss of critical information necessary for accurate predictions.

Such an outcome implies that the variability between the different location datasets contains valuable information that is pertinent to the predictive modeling of oxide compositions.
Averaging these datasets seems to dilute these nuances, leading to a generalized representation of the samples that does not capture the specificities required for precise predictions.
This is particularly evident in the case of \ce{SiO2} and \ce{Al2O3}, where the RMSE more than doubled when using aggregated datasets, highlighting a significant degradation in model performance.

\subsection{Experiment: PLS Automated Outlier Removal}\label{sec:experiment_pls_automated_outlier_removal}
When \citet{cleggRecalibrationMarsScience2017} developed the PLS1-SM models, outliers were removed based on manual inspection of the leverage and spectral residuals plots.
We instead chose to automate this based on the reasons described in Section~\ref{sec:methodology_outlier_removal}.
As such, examining the performance implications of completely omitting outlier removal is worthwhile.
Furthermore, the experiment is justified given the substantial efforts dedicated to developing the ChemCam calibration dataset as mentioned in Section~\ref{sec:ica_data_preprocessing}, which implies a minimal presence of significant outliers.

Table \ref{tab:pls1_sm_no_outlier_rmses} shows the RMSEs for the PLS1-SM replica model with and without automated outlier removal.
The RMSEs are very similar, which indicates that the automated outlier removal does not have a significant impact on the model performance.
This is expected based on the aforementioned efforts that went into developing the ChemCam calibration dataset, as described in Section~\ref{sec:ica_data_preprocessing}.

\begin{table}[H]
\centering
\begin{tabular}{lll}
\hline
Element    & Baseline & Without outlier removal \\
\hline
\ce{SiO2}  & \textbf{5.81}     & \textbf{5.81}                    \\
\ce{TiO2}  & \textbf{0.47}     & \textbf{0.47}                    \\
\ce{Al2O3} & 1.94              & \textbf{1.91}                    \\
\ce{FeO_T} & \textbf{4.35}     & \textbf{4.35}                    \\
\ce{MgO}   & \textbf{1.17}     & \textbf{1.17}                    \\
\ce{CaO}   & \textbf{1.43}     & 1.44                    \\
\ce{Na2O}  & \textbf{0.66}     & 0.67                    \\
\ce{K2O}   & 0.72              & \textbf{0.70}                    \\
\hline
\end{tabular}
\caption{RMSEs for the PLS1-SM model without automated outlier removal.}
\label{tab:pls1_sm_no_outlier_rmses}
\end{table}

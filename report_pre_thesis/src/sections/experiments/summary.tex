\subsection{Summarizing the experiments}\label{sec:experiments_summary}
In this chapter, we have conducted a series of experiments to evaluate the MOC pipeline, each revealing important insights into its various components.

In Section~\ref{sec:experiment_pls_automated_outlier_removal} we investigated the efficacy of our automated outlier removal. 
We demonstrated that our approach did not yield any significant benefits when compared to no outlier removal.

Additionally, in Section~\ref{sec:experiment_pls_fixed_thresholds} we experimented with fixed threshold values for outlier removal.
This experiment further validated that our outlier removal process did not improve model performance despite using a more conservative approach.

%Needs to reflect our actual results: Our application of the MAD method for outlier removal in the ICA phase offered a comparative perspective against traditional approaches, setting a benchmark for future enhancements. The evaluation of ICA’s performance with aggregated datasets illuminated the trade-offs between data representativeness and information loss, emphasizing the importance of diverse datasets.
%The comparative analysis with alternative models provided a broader understanding of the MOC pipeline's performance, identifying areas for future improvement.

Finally, in Section~\ref{sec:experiment_other_models} we tested the performance of two different types of models, namely XGBoost and an ANN. 
Here we demonstrated that both approaches are promising alternatives, both yielding comparable or better predictions in almost all cases.
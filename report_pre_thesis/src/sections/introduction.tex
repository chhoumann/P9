\section{Introduction}\label{sec:introduction}
% MSL: What is the mission? Why?
In November 2011, NASA launched the Mars Science Laboratory (MSL) mission, which landed the Curiosity rover on Mars in August 2012 inside Gale Crater. Its purpose is to investigate the Martian climate and geology, as well as the potential for life on Mars.
Early in the mission, Curiosity found chemical and mineral evidence that Mars once had the conditions necessary to support life.\cite{chemcam_nasa_website}

The rover itself is about the size of a car.
It has various instruments and cameras, including a laser instrument called ChemCam (Chemistry and Camera), which is used to analyze the chemical composition of rocks and soil on Mars.
The primary interest are rocks that either formed in water or show indications of organic materials.\cite{chemcam_nasa_website}

% ChemCam: How does it contribute to the mission? How?
ChemCam is a remote-sensing laser instrument developed by NASA in collaboration with the French national space agency (CNES).
The instrument is used to gather Laser-Induced Breakdown Spectroscopy (LIBS) data from geological samples on Mars. The data itself consists of a series of spectral readings, each forming a spectrum. These spectra represent the emitted light from a plasma created when the laser interacts with the target sample. Captured over a range of wavelengths, each spectrum is composed of various emission lines. Each emission line is associated with a specific element, and its intensity reflects the concentration of that element in the sample. Consequently, the collection of spectra serves as a complex, multi-dimensional fingerprint of the elemental composition of the examined geological formations.
This data is then used to determine the elemental composition of these samples.\cite{cleggRecalibrationMarsScience2017}

% These are computationally challenging problems because...
% Multicollinearity: In computational terms, multicollinearity represents a challenge for feature selection and model interpretability. Traditional regression algorithms struggle when variables (features) are highly correlated because they cannot easily isolate the contribution of each variable to the prediction (output). This hampers the model's ability to generalize well to new data and makes it difficult to understand which features are most important for the prediction task. Advanced techniques like ridge regression or feature extraction methods may be required to handle this.
% 
% Matrix Effects: These constitute a challenge in data preprocessing and normalization. In machine learning terms, matrix effects introduce a form of 'class imbalance' or 'data skew.' The model can be misled by the dominant features (in this case, spectral lines influenced by matrix effects) and fail to generalize well. Dealing with this requires sophisticated preprocessing techniques or specialized algorithms capable of handling imbalanced or skewed data.

% However, the interpretation of LIBS data poses significant computational challenges.
% First, a high degree of multicollinearity exists within the spectral data, rendering traditional linear analysis methods less effective.
% The multicollinearity arises due to the correlation among different spectral channels, influenced both by the multi-line emission characteristics of individual elements and by geochemical correlations between elements.
% Secondly, the complexity of LIBS spectra is increased by multiple interacting physical processes.
% These interactions, collectively referred to as 'matrix effects,' introduce variability into the emission line intensities independent of the elements' concentrations.
% Such variability complicates the direct interpretation of the spectra and poses challenges for computational models aiming for accurate elemental quantification.\cite{andersonImprovedAccuracyQuantitative2017}

The ChemCam team uses a model called the Multivariate Oxide Composition (MOC) model to predict the composition of major oxides in geological samples. As part of their preprocessing, they use various techniques to remove noise and outliers from the data.\cite{cleggRecalibrationMarsScience2017}

% -> This leads us to the predictioin of composition of major oxides
% There exists a model for this, but it has some limitations
The MOC model itself uses a series of submodels, each of which predicts the concentration of a single oxide. It combines the output of these with regression models based on ICA scores to predict the composition of major oxides in geological samples.
The model is trained on a calibration dataset consisting of LIBS data from 408 terrestrial rock samples, which were created in a laboratory to simulate the conditions of the Martian surface.\cite{cleggRecalibrationMarsScience2017}

% We narrow down to limitations that in any way inhibit the precision & accuracy of the model
% // _Any_ improvements to the model will be beneficial to the mission
% We want to investigate these limitations and propose improvements
Improving the predictions of the chemical compositions of LIBS data has the potential to advancing the Mars Science Laboratory's scientific objectives.
Enhanced accuracy in elemental quantification can give us more detailed information about Mars' geology and habitability, thereby amplifying the mission's overall scientific yield. 

% Consequently, there is a need to not only identify and examine the existing model's limitations but also to understand precisely why improvements would be beneficial and how they can be effectively implemented.

% As such, we aim to identify potential limitations and areas of improvement in the Multivariate Oxide Composition model's predictive accuracy and robustness in predicting the composition of major oxides in Martian geological samples using Earth-based calibration data.

Our study aims to identify and rectify limitations in the Multivariate Oxide Composition model used for analyzing LIBS data from Martian geological samples. By leveraging Earth-based calibration data, we aim to enhance predictive accuracy and robustness, thereby furthering the Mars Science Laboratory's scientific objectives.

% -- Introduction
% Introduce the problem and why it is interesting. The problem is the accurate and robustness of the MOC model.
% Our approach to the problem is to investigate the limitations of the current model and propose improvements.
% -- Methodology
% We'll do that by analyzing the data used to train the model and the model itself.
% In particular, we focus on aspects that may inhibit the robustness and accuracy of the model. That is, how accurately it predicts the composition of major oxides, as well as how well it generalizes to new data.
% This is done by first re-creating the current model and then reproducing the results of the original papers.
% After this, we'll investigate the limitations of the model and propose improvements.
% -- Results
% 
% -- Discussion
% 



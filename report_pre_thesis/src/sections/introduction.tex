\section{Introduction}\label{sec:introduction}
In November 2011, NASA launched the Mars Science Laboratory (MSL) mission, which landed the Curiosity rover on Mars in August 2012 inside Gale Crater. Its purpose is to investigate the Martian climate and geology, as well as the potential for life on Mars.
Early in the mission, Curiosity found chemical and mineral evidence that Mars once had the conditions necessary to support life.\cite{chemcamNasaWebsite}

The rover itself is about the size of a car.
It has various instruments and cameras, including a laser instrument called ChemCam (Chemistry and Camera), which is used to analyze the chemical composition of rocks and soil on Mars.
The primary interest are rocks that either formed in water or show indications of organic materials.\cite{chemcamNasaWebsite}

ChemCam is a remote-sensing laser instrument developed by NASA in collaboration with the French national space agency (CNES).
The instrument is used to gather Laser-Induced Breakdown Spectroscopy (LIBS) data from geological samples on Mars. The data itself consists of a series of spectral readings, each forming a spectrum. These spectra represent the emitted light from a plasma created when the laser interacts with the target sample. Captured over a range of wavelengths, each spectrum is composed of various emission lines. Each emission line is associated with a specific element, and its intensity reflects the concentration of that element in the sample. Consequently, the collection of spectra serves as a complex, multi-dimensional fingerprint of the elemental composition of the examined geological formations.
This data is then used to determine the elemental composition of these samples.\cite{cleggRecalibrationMarsScience2017}

LIBS is a versatile analytical tool with broad applicability across various fields. In environmental monitoring, its spectral data are effectively used with machine learning and statistical methods like partial least squares and neural networks for detecting and quantifying soil pollutants. In industrial contexts, it is also utilized for quality control processes involving metals and alloys\cite{huang_progress_2023}.

The ChemCam team uses an analytical system called the \textit{Multivariate Oxide Composition} (MOC) model to predict the composition of major oxides based on LIBS data from geological samples. 
The system is comprised of various components, each responsible for a specific task in predicting the composition of major oxides in geological samples.
In this context, a 'component' refers to a distinct, isolatable process or model affecting the system's overall predictive function.
The system utilizes a series of submodels to predict individual oxide concentrations, which are then combined with regression models that are based on ICA scores.
As part of their preprocessing, they use various techniques to remove noise and outliers from the data.\cite{cleggRecalibrationMarsScience2017}
The model is trained on a calibration dataset consisting of LIBS data from 408 terrestrial rock samples, simulated to mimic Martian conditions\cite{cleggRecalibrationMarsScience2017}.

The Mars Science Laboratory has made notable progress in planetary exploration, largely relying on models like the Multivariate Oxide Composition (MOC) to interpret Laser-Induced Breakdown Spectroscopy (LIBS) data from Martian geological samples.
Despite its utility, the existing MOC model shows limitations in predictive accuracy and robustness.
Enhancing the predictive accuracy and robustness of the MOC model is crucial for achieving more reliable composition predictions, thereby furthering the scientific objectives of the Mars Science Laboratory in understanding Martian geology and potential habitability.
Accuracy, in this context, is measured as Root Mean Squared Error (RMSE).
Robustness refers to the model's ability to handle the variations in the data.
We use a term 'matrix effects' as a catch-all term for any effect that can cause the intensity of emission lines from an element to vary independent of that element's concentration.
The complexity of LIBS spectra is increased by multiple interacting physical processes.
These interactions introduce variability into the emission line intensities independent of the elements' concentrations.
Such variability complicates the direct interpretation of the spectra and poses challenges for computational models aiming for accurate elemental quantification.\cite{andersonImprovedAccuracyQuantitative2017}

\textit{In this work, we aim to solve the problem of identifying and proposing improvements to the specific components of the current MOC model that limit its predictive accuracy and robustness against matrix effects.}

The remainder of this paper is organized as follows:
Section~\ref{sec:background} sets the context, while Section~\ref{sec:related_works} reviews existing literature.
The problem is formalized in Section~\ref{sec:definition}.
We analyze ChemCam sample data in Section~\ref{subsec:data_analysis} and present our methodology in Section~\ref{sec:methodology}.
Results are reported in Section~\ref{sec:results} and discussed in Section~\ref{sec:discussion}.
We conclude on our findings in Section~\ref{sec:conclusion} and suggest future research directions in Section~\ref{sec:recommendations_for_future_work}.

\section{Introduction}\label{sec:introduction}
% MSL: What is the mission? Why?
In November 2011, NASA launched the Mars Science Laboratory (MSL) mission, which landed the Curiosity rover on Mars in August 2012 inside Gale Crater. Its purpose is to investigate the Martian climate and geology, as well as the potential for life on Mars.
Early in the mission, Curiosity found chemical and mineral evidence that Mars once had the conditions necessary to support life.\cite{chemcamNasaWebsite}

The rover itself is about the size of a car.
It has various instruments and cameras, including a laser instrument called ChemCam (Chemistry and Camera), which is used to analyze the chemical composition of rocks and soil on Mars.
The primary interest are rocks that either formed in water or show indications of organic materials.\cite{chemcamNasaWebsite}

% ChemCam: How does it contribute to the mission? How?
ChemCam is a remote-sensing laser instrument developed by NASA in collaboration with the French national space agency (CNES).
The instrument is used to gather Laser-Induced Breakdown Spectroscopy (LIBS) data from geological samples on Mars. The data itself consists of a series of spectral readings, each forming a spectrum. These spectra represent the emitted light from a plasma created when the laser interacts with the target sample. Captured over a range of wavelengths, each spectrum is composed of various emission lines. Each emission line is associated with a specific element, and its intensity reflects the concentration of that element in the sample. Consequently, the collection of spectra serves as a complex, multi-dimensional fingerprint of the elemental composition of the examined geological formations.
This data is then used to determine the elemental composition of these samples.\cite{cleggRecalibrationMarsScience2017}

% These are computationally challenging problems because...
% Multicollinearity: In computational terms, multicollinearity represents a challenge for feature selection and model interpretability. Traditional regression algorithms struggle when variables (features) are highly correlated because they cannot easily isolate the contribution of each variable to the prediction (output). This hampers the model's ability to generalize well to new data and makes it difficult to understand which features are most important for the prediction task. Advanced techniques like ridge regression or feature extraction methods may be required to handle this.
% 
% Matrix Effects: These constitute a challenge in data preprocessing and normalization. In machine learning terms, matrix effects introduce a form of 'class imbalance' or 'data skew.' The model can be misled by the dominant features (in this case, spectral lines influenced by matrix effects) and fail to generalize well. Dealing with this requires sophisticated preprocessing techniques or specialized algorithms capable of handling imbalanced or skewed data.

% However, the interpretation of LIBS data poses significant computational challenges.
% First, a high degree of multicollinearity exists within the spectral data, rendering traditional linear analysis methods less effective.
% The multicollinearity arises due to the correlation among different spectral channels, influenced both by the multi-line emission characteristics of individual elements and by geochemical correlations between elements.
% Secondly, the complexity of LIBS spectra is increased by multiple interacting physical processes.
% These interactions, collectively referred to as 'matrix effects,' introduce variability into the emission line intensities independent of the elements' concentrations.
% Such variability complicates the direct interpretation of the spectra and poses challenges for computational models aiming for accurate elemental quantification.\cite{andersonImprovedAccuracyQuantitative2017}

The ChemCam team uses a model called the Multivariate Oxide Composition (MOC) model to predict the composition of major oxides in geological samples. As part of their preprocessing, they use various techniques to remove noise and outliers from the data.\cite{cleggRecalibrationMarsScience2017}

% -> This leads us to the predictioin of composition of major oxides
% There exists a model for this, but it has some limitations
The MOC model itself uses a series of submodels, each of which predicts the concentration of a single oxide. It combines the output of these with regression models based on ICA scores to predict the composition of major oxides in geological samples.
The model is trained on a calibration dataset consisting of LIBS data from 408 terrestrial rock samples, which were created in a laboratory to simulate the conditions of the Martian surface.\cite{cleggRecalibrationMarsScience2017}

% We narrow down to limitations that in any way inhibit the precision & accuracy of the model
% // _Any_ improvements to the model will be beneficial to the mission
% We want to investigate these limitations and propose improvements
The Mars Science Laboratory has made notable progress in planetary exploration, largely relying on models like the Multivariate Oxide Composition (MOC) to interpret Laser-Induced Breakdown Spectroscopy (LIBS) data from Martian geological samples.
Despite its utility, the existing MOC model shows limitations in predictive accuracy and robustness.
Enhancing the predictive accuracy and robustness of the MOC model is crucial for achieving more reliable composition predictions, thereby furthering the scientific objectives of the Mars Science Laboratory in understanding Martian geology and potential habitability.
This research will identify and rectify the limitations of the existing MOC model, focusing on data obtained from Martian geological samples, to improve both its predictive accuracy and robustness.

The remainder of this paper is organized as follows:
Section~\ref{sec:background} sets the context, while Section~\ref{sec:related_works} reviews existing literature.
The problem is formalized in Section~\ref{sec:definition}.
We analyze ChemCam sample data in Section~\ref{subsec:data_analysis} and present our methodology in Section~\ref{sec:methodology}.
Results are reported in Section~\ref{sec:results} and discussed in Section~\ref{sec:discussion}.
We conclude on our findings in Section~\ref{sec:conclusion} and suggest future research directions in Section~\ref{sec:recommendations_for_future_work}.

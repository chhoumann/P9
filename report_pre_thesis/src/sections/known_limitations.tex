% MULTICOLINEARITY & MATRIX EFFECTS - EXPLANATION
% Multicollinearity: In computational terms, multicollinearity represents a challenge for feature selection and model interpretability. Traditional regression algorithms struggle when variables (features) are highly correlated because they cannot easily isolate the contribution of each variable to the prediction (output). This hampers the model's ability to generalize well to new data and makes it difficult to understand which features are most important for the prediction task. Advanced techniques like ridge regression or feature extraction methods may be required to handle this.
%
% Matrix Effects: These constitute a challenge in data preprocessing and normalization. In machine learning terms, matrix effects introduce a form of 'class imbalance' or 'data skew.' The model can be misled by the dominant features (in this case, spectral lines influenced by matrix effects) and fail to generalize well. Dealing with this requires sophisticated preprocessing techniques or specialized algorithms capable of handling imbalanced or skewed data.

% MULTICOLINEARITY & MATRIX EFFECTS - PROSE
The interpretation of LIBS data poses significant computational challenges.
First, a high degree of multicollinearity exists within the spectral data, rendering traditional linear analysis methods less effective.
The multicollinearity arises due to the correlation among different spectral channels, influenced both by the multi-line emission characteristics of individual elements and by geochemical correlations between elements\cite{andersonImprovedAccuracyQuantitative2017}.
Secondly, the complexity of LIBS spectra is increased by multiple interacting physical processes - the aforementioned \textit{matrix effects}.
These interactions introduce variability into the emission line intensities independent of the elements' concentrations.
It is possible to partially account for these effects by using multivariate algorithms that make use of the information contained in the entire spectrum, rather than individual lines\cite{andersonImprovedAccuracyQuantitative2017}.
Such variability complicates the direct interpretation of the spectra and poses challenges for computational models aiming for accurate elemental quantification.

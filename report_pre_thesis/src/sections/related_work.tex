\section{Related work}\label{sec:related_works}
% The Literature Review provides an overview of existing work related to your study.
% - Purpose: To show how your work fits into the existing body of knowledge, identify gaps, and justify your research.
% - Subsections: May be organized by methodologies, chronological developments, or thematic topics.
% - Best Practices: Maintain a critical perspective, identifying not just what other works have found but also their limitations.

% Introduction to the Field
% - Overview of the research area
% Existing Solutions
% - Discussion of current methods, algorithms, or systems
% Gaps in Current Research
% - Identified shortcomings in existing work
% Theoretical Frameworks
% - Any theories or models that are foundational to the research

Quantitative analysis of LIBS spectra is a well-studied problem.
The LIBS community has developed a variety of techniques to address this problem, including calibration-free LIBS (CF-LIBS), partial least squares (PLS) regression, independent component analysis (ICA), and artificial neural networks (ANNs).

There is no agreed-upon state of the art method for quantitative analysis of LIBS spectra, as different methods have been shown to perform better than others depending on the application.

% Connect the above introduction with the background section because it explains what Anderson et al. did and why they did it.

Since then, there have been several studies that have tried to address the limitations of the approach used by Anderson et al. 2017.

Takashi et al. reviewed evaluated methods for correcting matrix effects and self-absorption in the quantitative analysis of solid compositions via LIBS.
They found that while traditional calibration methods for LIBS face challenges with complex in-situ samples, CF-LIBS is limited by the need for prior sample knowledge.
Multivariate analysis techniques like PCR and PLS handle matrix effects well but struggle with non-linearities like self-absorption.
Despite limited research into the use of ANNs for quantitative analysis of solid samples, they found that ANN showed potential to overcome these issues because ANNs are able to learn the non-linear relationship between the LIBS spectra and the composition of the sample.
However, ANN is limited by the need for a large number of training samples.
They conclude that simpler matrices favor theory-based methods, and more complex samples benefit from flexible statistical approaches, with computational advancements promising further improvements in LIBS analysis.




\begin{itemize}
	\item What did they do with PLS in Anderson et al. 2017 and why? (reference back to christian's section!)
	\item What did they do with ICA in Forni et al. 2013 and why? (reference back to christian's section!)
	\item Which limitations do they face? (e.g. matrix effects).
	\item Related work since 2017 that has tried to address these limitations/suggest improvements.
	\begin{itemize}
		\item Kate et al.'s work
		\item CNN approach
		\item Elastic net
		\item Larger dataset
		\item Compensation of matrix effects (https://www.scopus.com/record/display.uri?eid=2-s2.0-85031935321&origin=resultslist&sort=plf-f&cite=2-s2.0-85010886895&src=s&nlo=&nlr=&nls=&imp=t&sid=80e7a93fa62b9ee1c75d5f3366219938&sot=cite&sdt=a&sl=0&relpos=63&citeCnt=104&searchTerm=)
	\end{itemize}
	\item Other fields
\end{itemize}
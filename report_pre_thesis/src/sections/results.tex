\section{Results}\label{sec:results}
In this section, we detail the outcomes of our MOC pipeline replica and the experiments described in Section~\ref{sec:methodology}.
The performance of our replica serves as baseline for comparison with the original MOC pipeline and as a basis for assessing experimental results.

\subsection{MOC Results}
Table \ref{tab:results_rmses} shows the RMSEs of the original and our replicas of the PLS1-SM, ICA, and MOC models.

% TODO: Comment on the results
% TODO: Test if there is a statistically significant difference between the original and replica models

\begin{table*}[t]
\centering
\begin{tabular*}{\textwidth}{@{\extracolsep{\fill}}lcccccc}
\hline
Element    & PLS1-SM (original) & PLS1-SM (replica) & ICA (original) & ICA (replica) & MOC (original) & MOC (replica) \\
\hline
\ce{SiO2}  & 3.39               & 5.71              & 8.31           & 9.89          & 5.30           & 6.97 \\
\ce{TiO2}  & 1.33               & 0.51              & 1.44           & 0.55          & 1.03           & 0.52 \\
\ce{Al2O3} & 2.97               & 1.96              & 4.77           & 4.25          & 3.47           & 3.66 \\
\ce{FeO_T} & 2.71               & 4.13              & 5.17           & 3.74          & 2.31           & 3.45 \\
\ce{MgO}   & 1.28               & 1.11              & 4.08           & 3.13          & 2.21           & 1.75 \\
\ce{CaO}   & 0.35               & 1.51              & 3.07           & 3.58          & 2.72           & 5.75 \\
\ce{Na2O}  & 1.09               & 0.65              & 2.29           & 3.02          & 0.62           & 2.38 \\
\ce{K2O}   & 0.62               & 0.70              & 0.98           & 1.40          & 0.82           & 0.91 \\
\hline
\end{tabular*}
\caption{RMSE of the original and our replicas of the PLS1-SM, ICA, and MOC models.}
\label{tab:results_rmses}
\end{table*}

As can be seen in Table~\ref{tab:results_rmses} we have presented a pairwise comparison between the errors obtained by the original models and our own.
From the table it is clear that the errors obtained from our ICA and MOC replication are similar to the originals, while the errors obtained from our PLS1-SM replication are quite different.
Performing a T-test between the models confirm these observations:
\begin{itemize}
    \item p-value for PLS1-SM: 3.44\%
    \item p-value for ICA: 85.57\%
    \item p-value for MOC: 9.13\%
\end{itemize}



\subsection{ANN Results}
Table \ref{tab:results_ann} shows the RMSEs of the original MOC model and our ANN model.

\begin{table}[t]
\centering
\begin{tabular}{lcc}
\hline
Element    & MOC (original) & ANN \\
\hline
\ce{SiO2}  & 5.30           & \textbf{4.56} \\
\ce{TiO2}  & 1.03           & \textbf{0.57} \\
\ce{Al2O3} & 3.47           & \textbf{2.47} \\
\ce{FeO_T} & \textbf{2.31}  & 2.45          \\
\ce{MgO}   & 2.21           & \textbf{1.35} \\
\ce{CaO}   & 2.72           & \textbf{1.50} \\
\ce{Na2O}  & \textbf{0.62}  & 3.05          \\
\ce{K2O}   & \textbf{0.82}  & 0.93          \\
\hline
\end{tabular}
\caption{RMSE of the original MOC model and our ANN model.}
\label{tab:results_ann}
\end{table}


% Analysis / Experimental / Empirical Evaluation (the analysis, comparing our method to SoA or a baseline, ablation study)
% \begin{itemize}
%     \item System used (for repeatability)
%     \item Datasets / people used
%     \item Metrics used
%     \item Results: a list of potential improvements or alterations that would enhance the model's predictive accuracy and robustness, as well as explanations as to why these improvements would be beneficial.
% \end{itemize}

% Data Presentation
% - Tables, graphs, and any other illustrations of the collected data
% Analysis
% - Interpreting the data in terms of the problem definition and research objectives
% Statistical Tests
% - Any statistical methods applied to the data to support the analysis
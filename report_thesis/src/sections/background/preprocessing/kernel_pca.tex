\subsubsection{Kernel PCA}
We provide a brief overview of the \gls{kernel-pca} algorithm based on \citet{learningwithkernels}.
\gls{kernel-pca} is an extension of traditional \gls{pca} designed to handle nonlinear relationships among data points.
The core idea behind \gls{kernel-pca} is to map data into a higher-dimensional space using a kernel function, a technique known as the kernel trick.
This mapping enables linear separation of data points in the higher-dimensional space, even if they are not linearly separable in the original space.

Similar to \gls{pca}, as described in Section~\ref{subsec:pca}, the goal of \gls{kernel-pca} is to extract the principal components of the data.
Unlike \gls{pca}, \gls{kernel-pca} does not compute the covariance matrix of the data directly, as this is often infeasible for high-dimensional datasets.
Instead, \gls{kernel-pca} leverages the kernel trick to compute the similarities between data points directly in the original space using a kernel function.
This kernel function implicitly computes the dot product in the higher-dimensional feature space without explicitly mapping the data points into that space.
That way, \gls{kernel-pca} can capture nonlinear relationships among data points without explicitly transforming them into a higher-dimensional space.

By using pairwise similarities to construct a kernel matrix, also referred to as a Gram matrix, \gls{kernel-pca} can perform eigenvalue decomposition.
This process allows for the extraction of principal components in the feature space, similar to the approach used in regular \gls{pca}.
However, in \gls{kernel-pca}, the eigenvalue decomposition is performed on the kernel matrix rather than the covariance matrix, resulting in the principal components.
These principal components are nonlinear combinations of the original data points, enabling the algorithm to capture complex relationships among data points that are not linearly separable in the original space.
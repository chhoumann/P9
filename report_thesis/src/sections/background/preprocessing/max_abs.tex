\subsubsection{Max Absolute Scaler}
Max absolute scaling is a normalization technique that scales each feature individually so that the maximum absolute value of each feature is 1.
This results in the data being normalized to a range between -1 and 1.
The formula for max absolute scaling is given by:

$$
x'_i = \frac{x_i}{\max(|\mathbf{x}|)},
$$

where $x_i$ is the original feature value, $\max(|\mathbf{x}|)$ is the maximum absolute value of the feature vector $\mathbf{x}$, and $x'_i$ is the normalized feature value.
This scaling method is particularly useful for data that has been centered at zero or is sparse, as max absolute scaling does not alter the mean of the data.
Additionally, it preserves the sparsity of the data by ensuring that zero entries remain zero, thereby not introducing any non-zero values~\cite{Vasques2024}.
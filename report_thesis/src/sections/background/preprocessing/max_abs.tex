\subsubsection{Max Absolute Scaler}
Max absolute scaling is a normalization technique that scales each feature individually so that the maximum absolute value of each feature is 1.
This results in the data being normalized to a range between -1 and 1.
The formula for max absolute scaling is given by:

$$
x'_i = \frac{x_i}{\max(|\mathbf{x}|)},
$$

where $x_i$ is the original feature value, $\max(|\mathbf{x}|)$ is the maximum absolute value of the feature vector $\mathbf{x}$, and $x'_i$ is the normalized feature value.
This scaling method is useful for data that has been centered at zero or data that is sparse, as max absolute scaling does not center the data.
It maintains the sparsity of the data by not introducing non-zero values in the zero entries of the data~\cite{Vasques2024}.
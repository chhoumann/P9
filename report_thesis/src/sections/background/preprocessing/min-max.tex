\subsubsection{Min-Max Normalization}\label{subsec:min-max}
Min-Max normalization rescales the range of features to a specific range $[a, b]$, where $a$ and $b$ represent the new minimum and maximum values, respectively.
The goal is to normalize the range of the data to a specific scale, typically 0 to 1.
The Min-Max normalization of a feature vector $\mathbf{x}$ is given by:

$$
x'_i = \frac{x_i - \min(\mathbf{x})}{\max(\mathbf{x}) - \min(\mathbf{x})}(b - a) + a,
$$

where $x_i$ is the original value, $\min(\mathbf{x})$ and $\max(\mathbf{x})$ are the minimum and maximum values of the feature vector $\mathbf{x}$, respectively, and $x'_i$ is the normalized feature value.

This type of normalization is beneficial because it ensures that each feature contributes equally to the analysis, regardless of its original scale~\cite{dataminingConcepts}.
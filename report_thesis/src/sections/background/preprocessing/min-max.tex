\subsubsection{Min-Max Normalization}\label{subsec:min-max}
Min-max normalization rescales the range of features to $[0, 1]$ or $[a, b]$, where $a$ and $b$ represent the new minimum and maximum values, respectively.
The goal is to normalize the range of the data to a specific scale, typically 0 to 1.
Mathematically, min-max normalization is defined as:
$$
	v' = \frac{v - \min(F)}{\max(F) - \min(F)} \times (b - a) + a,
$$
where $v$ is the original value, $\min(F)$ and $\max(F)$ are the minimum and maximum values of the feature $F$, respectively.

This type of normalization is beneficial because it ensures that each feature contributes equally to the analysis, regardless of its original scale.
\subsubsection{Power Transformation}
Power transformations are a class of mathematical functions used to stabilize variance and make data more closely approximate a normal distribution.
They are particularly useful in statistical modeling and data analysis to meet the assumptions of linear models.

One of the first influential power transformation techniques is the Box-Cox power transform, introduced by \citet{BoxAndCox} in 1964.
This is defined for positive data and is aimed at normalizing data or making it more symmetric.
For a feature value $x$, the Box-Cox transformation is defined as:

$$
\psi^{\text{BC}}(\lambda, x) =
\begin{cases}
\frac{x^\lambda - 1}{\lambda}, & (\lambda \neq 0) \\
\log(x), & (\lambda = 0)
\end{cases},
$$

where $\lambda$ is the transformation parameter.
$\lambda$ determines the extend and nature of the transformation, where positive values of $\lambda$ apply a power transformation and $\lambda = 0$ applies a logarithmic transformation.

To overcome the limitations of the Box-Cox transformation, \citet{YeoJohnson} introduced a new family of power transformations that can handle both positive and negative values.
The Yeo-Johnson power transformation is defined as:

$$
\psi(\lambda, x) =
\begin{cases}
\frac{(x + 1)^\lambda - 1}{\lambda} & (x \geq 0, \lambda \neq 0) \\
\log(x + 1) & (x \geq 0, \lambda = 0) \\
- \frac{(-x + 1)^{2 - \lambda} - 1}{2 - \lambda} & (x < 0, \lambda \neq 2) \\
-\log(-x + 1) & (x < 0, \lambda = 2)
\end{cases}
$$

For non-negative values, the Yeo-Johnson transformation simplifies to the Box-Cox transformation, making them equivalent in this context.
The key benefit of the Yeo-Johnson transformation is its ability to handle any real number, making it a robust choice for transforming data to achieve approximate normality or symmetry.
This property is particularly beneficial for preparing data for statistical analyses and machine learning models that require normally distributed input data.
\subsubsection{Z-score Normalization}
Z-score normalization, also standardization, transforms data to have a mean of zero and a standard deviation of one.
This technique is useful when the actual minimum and maximum of a feature are unknown or when outliers may significantly skew the distribution.
The formula for Z-score normalization is given by:

$$
v' = \frac{v - \overline{F}}{\sigma_F},
$$

where $v$ is the original value, $\overline{F}$ is the mean of the feature $F$, and $\sigma_F$ is the standard deviation of $F$.
By transforming the data using the Z-score, each value reflects its distance from the mean in terms of standard deviations.
Z-score normalization is particularly advantageous in scenarios where data features have different units or scales, or when preparing data for algorithms that assume normally distributed inputs~\cite{dataminingConcepts}.
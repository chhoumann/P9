\section{Introduction}\label{sec:introduction}

NASA has been studying the Martian environment for decades through a series of missions, including the Viking missions~\cite{marsnasagov_vikings}, the \gls{mer} mission~\cite{marsnasagov_observer, marsnasagov_spirit_opportunity}, and the \gls{msl} mission~\cite{marsnasagov_msl}, each building on the knowledge gained from the previous ones.
Today, the rovers exploring Mars are equipped with sophisticated instruments for analyzing the chemical composition of Martian soil in search of past life and habitable environments.

Part of this research is facilitated through interpretation of spectral data gathered  by \gls{libs} instruments.
\gls{libs} works by firing a high-powered laser at soil samples, creating a plasma that emits light captured by spectrometers and analyzed for its elemental composition through the use of machine learning models.
Scientists and researchers at NASA uses the spectral data to inform their understanding of Mars' geology through the identification of major oxides in the soil samples.
To this end, various machine learning models have been developed to predict the composition of major oxides in the sample, including \glspl{cnn}~\cite{yang_laser-induced_2022, yangConvolutionalNeuralNetwork2022}, \gls{svr}~\cite{rezaei_dimensionality_reduction}, and hybrid models like \gls{df}-\gls{k-elm}~\cite{song_DF-K-ELM} that incorporate domain knowledge to enhance model interpretability and performance.
However, the high dimensionality and multicollinearity of the spectral data remains a significant challenge for these models.

A detailed study into advanced machine learning models and dimensionality reduction techniques is therefore crucial for improving predictions in these applications.

% Placeholder
In this work, we propose a novel approach for predicting major oxide compositions in Martian soil samples from LIBS spectral data. Our method combines advanced dimensionality reduction techniques with state-of-the-art machine learning models to effectively handle the high-dimensional and multicollinear nature of the data. 
By \textit{...}
Through extensive experiments on real Martian LIBS data, we demonstrate the superior performance of our approach compared to existing methods in terms of both prediction accuracy and computational efficiency.


In this work, we show that... \textit{results of the study will be added here after the paper is written.}

Our experimental results demonstrate that... \textit{results of the experiment will be added here after the paper is written.}

Our key contributions are as follows:
\begin{itemize}
	\item \textit{List contributions here.}
\end{itemize}


The remainder of this paper is organized as follows:
\textit{Structure of the paper will be added here after the paper is written.}
% TODO: Describe the results, experiments, contributions and structure of the paper.
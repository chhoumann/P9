\section{Introduction}\label{sec:introduction}
The NASA Viking missions in the 1970s were the first to successfully land on Mars, aiming to determine if life existed on the planet.
One experiment suggested the presence of life, but the results were ambiguous and inconclusive, and NASA was unable to repeat the experiment.
Nevertheless, these missions were deemed a monumental success and advanced our knowledge of the Martian envi-\\ronment.\cite{marsnasagov_vikings}

Leveraging the knowledge gained from the Viking missions, NASA launched the \gls{mer} mission in 2003 to investigate whether Mars ever had the conditions to support life as we know it.
The mission landed two rovers, Spirit and Opportunity, on Mars in January 2004, and they quickly discovered clear evidence that water once flowed on Mars.
However, since water alone is not enough to support life, the next objective was to search for organic material as well.\cite{marsnasagov_observer, marsnasagov_spirit_opportunity}

The Curiosity rover landed on Mars in August 2012 inside Gale Crater as part of the \gls{msl} mission with this very purpose.
Thanks to its sophisticated equipment, Curiosity was able to find evidence of past habitable environments on Mars based on chemical and mineral findings early in its mission.\cite{marsnasagov_chemcam}

One of the instruments aboard the rover is the \gls{chemcam} instrument, which is a remote-sensing laser instrument used to gather \gls{libs} data from geological samples on Mars.
\gls{libs} is a technique that enables rapid analysis by using a laser to ablate and remove surface contaminants to expose the underlying material and generate a plasma plume from the now-exposed sample material\cite{wiensChemcam2012}.
This plasma plume emits light that is captured through three distinct spectrometers to collect a series of spectral readings.
These spectra consist of emission lines that can be associated with the concentration of a specific element, and their intensity reflects the concentration of that element in the sample.
Consequently, a spectra serves as a complex, multi-dimensional fingerprint of the elemental composition of the examined geological samples.\cite{cleggRecalibrationMarsScience2017}

Analyzing \gls{libs} data is computationally challenging due to high multicollinearity within spectral data, which diminishes the effectiveness of traditional linear analysis.
The multicollinearity, which stems from correlations among spectral channels and elemental emission characteristics, complicates data interpretation.
Additionally, \textit{matrix effects} in \gls{libs} spectra arise when various physical interactions cause emission line intensities to change without a corresponding shift in the element's actual concentration
This phenomenon introduces variability that complicates the straightforward interpretation of spectral data, challenging the accuracy of computational models tasked with predicting elemental composition.\cite{andersonImprovedAccuracyQuantitative2017}

For analyzing Martian geological samples, the \gls{chemcam} team currently uses the \gls{moc} model\cite{cleggRecalibrationMarsScience2017}.
This model integrates \gls{pls} and \gls{ica} to predict the composition of major oxides.
Though the MOC model has proven useful, it suffers from limitations in predictive accuracy and robustness.
In \citet{p9_paper}, we created a replica of the MOC model and identified which components were responsible for these limitations.
Through a series of comparative experiments, we showed that the model selection was the primary cause of these limitations, and we showed how both \gls{ann} and \gls{gbr} methods could be used to improve the model's predictive accuracy and robustness.

This is further underscored by work from the SuperCam team.
In 2021, the Perseverance rover landed on Mars, equipped with the SuperCam instrument, which is the successor to the \gls{chemcam} instrument.
As part of the ongoing work to support the SuperCam instrument, \citet{andersonPostlandingMajorElement2022} experimented with various machine learning models to predict the composition of major oxides in geological samples using the SuperCam \gls{libs} calibration dataset.
While the team decided to retain \gls{pls} for analyzing certain oxides, \gls{ica} was entirely discontinued.
Instead, models based on \gls{gbr}, \gls{rf}, and \gls{lasso} were selected for other oxides.
This decision reinforces our finding that \gls{ica} regression models fall short in accurately predicting the composition of major oxides in geological samples.
Consistent with our observations, \gls{gbr} was also identified as a high-performing model in their analyses.

However, there remains considerable uncertainty about which machine learning techniques best predict the composition of major oxides in Martian geological samples using \gls{libs} data.
This underscores the importance of a detailed study into advanced machine learning models for improving predictions in these applications.

In addition, the high dimensionality of \gls{libs} data poses a significant challenge for computational models.
Therefore, techniques that reduce the dimensionality of the data are crucial for mitigating the effects of multicollinearity and enhancing the model's ability to discern the underlying patterns within the spectral data.

\textit{In this work, we aim to investigate the application of dimensionality reduction techniques and advanced machine learning models to predict the composition of major oxides in Martian geological samples using \gls{libs} data.}

The remainder of this paper is organized as follows:
\textit{Structure of the paper will be added here after the paper is written.}
% TODO: Describe the structure of the paper here.
\section{Introduction}\label{sec:introduction}
NASA has been studying the Martian environment for decades through a series of missions, including the Viking missions~\cite{marsnasagov_vikings}, the \gls{mer} mission~\cite{marsnasagov_observer, marsnasagov_spirit_opportunity}, and the \gls{msl} mission~\cite{marsnasagov_msl}, each building on the knowledge gained from the previous ones.
Today, the rovers exploring Mars are equipped with sophisticated instruments for analyzing the chemical composition of Martian soil in search of past life and habitable environments.

Part of this research is facilitated through interpretation of spectral data gathered by \gls{libs} instruments, which fire a high-powered laser at soil samples to create a plasma.
The emitted light is captured by spectrometers and analyzed using machine learning models to assess the presence and concentration of certain major oxides, informing NASA's understanding of Mars' geology~\cite{cleggRecalibrationMarsScience2017}.

However, predicting major oxide compositions from \gls{libs} data still presents significant computational challenges.
These include the high dimensionality and non-linearity of the data, compounded by issues of multicollinearity and matrix effects~\cite{andersonImprovedAccuracyQuantitative2017}.
Such effects can cause the intensity of emission lines from an element to vary independently of that element's concentration, introducing unknown variables that complicate the analysis.
Furthermore, the high cost of data collection often results in small datasets, exacerbating the difficulty of building accurate and robust models.

Previous work has aimed to improve the prediction of major oxide compositions from \gls{libs} data by using regression techniques and dimensionality reduction with feature selection.
These methods have been used to enhance both the accuracy and interpretability of the prediction models.
Tailored approaches have also been developed, where different models are selected based on their performance with specific spectral characteristics~\cite{rezaei_dimensionality_reduction, andersonPostlandingMajorElement2022}.
Moreover, models incorporating physical principles have demonstrated improved accuracy by handling residuals that traditional models fail to explain~\cite{song_DF-K-ELM}.
However, predicting oxide compositions remains challenging due to the complex, nonlinear nature of \gls{libs} data.
This underscores the need for continued research into more adaptive and robust machine learning strategies to tackle these issues effectively.

This thesis aims to improve upon previous work in the field of \gls{libs} data analysis.
Our goal is to develop a machine learning pipeline that is tailored to the unique characteristics of \gls{libs} data, with the goal of achieving higher prediction accuracy and robustness.

To achieve these objectives, we build upon the baseline established in~\cite{p9_paper} and systematically explore ten different machine learning models. 
These models were identified and selected through a combination of extensive literature review and the consideration of unconventional approaches not typically covered in the \gls{libs}-based calibration literature.
The ten models fall into three categories: Ensemble learning models, linear and regularization models, and neural network models.
In addition to model exploration, we also investigate a selection of preprocessing techniques: scaling (including normalization), dimensionality reduction, and data transformation.
Specifically, we designed and implemented a framework for experimental analysis using the automated hyperparameter optimization framework Optuna~\cite{optuna_2019}.
We then used this framework to determine the most effective combinations of preprocessing methods and models for each regression target.
We began by identifying the most promising models from the literature, after which we evaluated various preprocessing techniques to understand their impact on model performance, selecting those that demonstrated the highest impact on improving the performance of each model.
Following preprocessing, we optimized the chosen models through hyperparameter tuning to ensure optimal performance tailored to the specific data characteristics of each oxide.
Once the best hyperparameters were identified, a stacking ensemble method was employed to create a meta learner for each oxide, significantly enhancing prediction accuracy and robustness beyond the capabilities of individual models.
% TODO: Superior performance needs to be replaced with actual numbers
Through extensive experiments on \gls{libs} data, we systematically assessed and demonstrated the superior performance of our approach compared to existing methods, focusing on significant improvements in prediction accuracy and robustness.

Our key contributions are as follows:
\begin{itemize}
    \item We develop a novel machine learning pipeline that demonstrates improved accuracy and robustness in predicting major oxide compositions in \gls{libs} data.
    \item We have developed a novel optimization approach and tool for tuning and evaluating machine learning models along with preprocessing techniques, providing a systematic and efficient method for selecting the best configuration.
    \item By outperforming existing methods, our approach has established new benchmarks for accuracy and robustness in \gls{libs} data analysis.
\end{itemize}


% TODO: Add remaining sections
The remainder of this paper is organized as follows:
Section~\ref{sec:background} provides background on the onoging Mars exploration missions, the \gls{libs} technique, and the baseline \gls{moc} model.
Section~\ref{sec:problem_definition} formally defines the problem addressed in this work.
Section~\ref{sec:methodology} describes our proposed methodology, including data preprocessing, dimensionality reduction, and machine learning models.
Section~\ref{sec:experiments} presents our experimental setup and results.
Finally, Section~\ref{sec:conclusion} concludes the paper and discusses future work.

In table \ref{tab:terms}, we provide a list of terms used throughout this paper.

\begin{table}
\centering
\begin{tabularx}{\columnwidth}{lX} % l for left, X for the cell that should be wrapped
\toprule
Term & Definition \\
\midrule
Sample & A physical specimen of rock, soil, or other material collected for scientific analysis.\\
Location & The specific point on a sample where a LIBS laser is targeted. There are typically multiple locations per sample. \\ 
Target & Refers to the variable that a machine learning model is trained to predict. \\
\bottomrule
\end{tabularx}  
\caption{Table of Terms}
\label{tab:terms}
\end{table}
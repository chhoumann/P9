\section{Methodology}\label{sec:methodology}
\textit{We will write an introduction to the methodology section here, as well as add more subsections in the future. Below is the first subsection describing the data normalization process and the reasons for choosing to only do Norm 3. Please let us know if the explanation and mathematical notation is clear.}

\subsection{Data Normalization}
For data normalization, \citet{cleggRecalibrationMarsScience2017} proposes two different normalization methods, Norm 1 and Norm 3.
Norm 1 normalizes the data across the entire spectrum, while Norm 3 normalizes the data across individual spectrometers' wavelength ranges.

Formally, Norm 1 is defined as:

\begin{equation}
	\tilde{X}_{i,j} = \frac{X_{i,j}}{\sum_{j=1}^{N} X_{i,j}},
\end{equation}

where,

\begin{itemize}
	\item $\tilde{X}_{i,j}$ is the normalized wavelength intensity for the $i$-th sample in the $j$-th channel.
	\item $X_{i,j}$ is the original wavelength intensity for the $i$-th sample in the $j$-th channel.
	\item $N$ is the number of channels in each spectrometer (for the \gls{chemcam} instrument, $N = 2048$).
\end{itemize}

Norm 3 is defined as:

\begin{equation}
	\tilde{X}_{j,k}^{(i)} = \frac{X_{j,k}}{\sum_{k'=N(i-1)+1}^{N \cdot i} X_{j,k'}},
\end{equation}

where:

\begin{itemize}
	\item $\tilde{X}_{j,k}^{(i)}$ is the normalized wavelength intensity for the $j$-th sample in the $k$-th channel on the $i$-th spectrometer.
	\item $X_{j,k}$ is the original wavelength intensity for the $j$-th sample in the $k$-th channel on the $i$-th spectrometer.
	\item $N$ is the number of channels in each spectrometer (for the \gls{chemcam} instrument, $N = 2048$)
	\item $i$ is the index of the spectrometer (for the \gls{chemcam} instrument, $i \in \{1, 2, 3\}$)
	\item $k'$ indexes over the channels of the $i$-th spectrometer, ensuring the $N$ channels of each spectrometer are normalized separately.
\end{itemize}

Following the approach taken by the SuperCam team, we opt to always normalize across individual spectrometers' wavelength ranges (Norm 3), rather than normalizing across the entire spectrum (Norm 1).
This decision is guided by the approach taken by the SuperCam team, where they do not normalize across the entire spectrum, but rather across individual spectrometers' wavelength ranges~\cite{andersonPostlandingMajorElement2022}.
\section{Experimental Design}\label{sec:methodology}
In this section, we outline the methodology used in this study to address the challenges identified in Section~\ref{sec:problem_definition}. Our objective was to identify the most promising machine learning models and preprocessing techniques proposed from the literature, as outlined in Section~\ref{sec:related-work}, for predicting major oxide compositions from \gls{libs} data.
Then, using this knowledge, develop a pipeline that utilizes the strengths of these models and preprocessing techniques to improve prediction accuracy and robustness of the predictions.
We first describe the datasets used, including their preparation and the method of splitting for model training. Next, we outline the preprocessing steps and the model selection process, followed by a detailed explanation of the experimental setup and evaluation metrics. Finally, we discuss our validation testing procedures and the approach taken to ensure unbiased final model evaluations.


\subsection{Data Preparation}
Similarly to our previous work \citet{p9_paper}, we used the publicly available \gls{ccs} data from NASA's \gls{pds}~\cite{PDSGeoscienceNode}.
\gls{ccs} refers to \gls{libs} data that has been through a series of preprocessing steps such as subtracting the ambient light background, noise removal and removing the electron continuum to derive data that is more suitable for quantitative analysis.
A comprehensive description of this preprocessing procedure is available in \citet{wiensPreflightCalibrationInitial2013}.

\begin{table*}[h]
\centering
\begin{tabular}{llllllll}
\toprule
     wave &         shot1 &         shot2 &  $\cdots$ &        shot49 &       shot50  & median        & mean          \\
\midrule
240.81100 & 6.4026649e+15 & 4.0429349e+15 & $\cdots$  & 1.7922483e+15 & 1.7126615e+15 & 1.9892956e+15 & 1.7561699e+15 \\
240.86501 & 3.8557462e+12 & 2.2923680e+12 & $\cdots$  & 1.1355429e+12 & 8.6930379e+11 & 7.8172542e+11 & 7.2805052e+11 \\
$\vdots$  & $\vdots$      & $\vdots$      & $\cdots$  & $\vdots$      & $\vdots$      & $\vdots$      & $\vdots$      \\
905.38062 & 1.8823427e+08 & 58500403.     & $\cdots$  & -8449286.6    & 8710775.0     & 4.0513312e+09 & 5.2188327e+09 \\
905.57349 & 1.9864713e+10 & 1.2956832e+10 & $\cdots$  & 1.9785415e+10 & 7.1994239e+09 & 1.1311150e+10 & 1.2201224e+10 \\
\bottomrule
\end{tabular}
\caption{Example of CCS data for a single location (from \citet{p9_paper})}
\label{tab:ccs_data_example}
\end{table*}

While the \gls{ccs} data is in a more suitable form for quantitative analysis, it still requires further preprocessing. Table~\ref{tab:ccs_data_example} shows an example of the \gls{ccs} data for a single location of a sample. This corresponds to shots ($s$) and wavelength ($\lambda$) of the Intensity Tensor \ref{matrix:intensity} for this location.
The initial five shots from each sample are excluded because they are usually contaminated by dust covering the sample, which is cleared away by the shock waves produced by the laser \cite{cleggRecalibrationMarsScience2017}.
The remaining 45 shots from each location are then averaged, yielding a single spectrum $s$ per location $l$ in the Averaged Intensity Tensor\ref{matrix:averaged_intensity}, resulting in a total of five spectra for each sample. 

At this stage, the data still contains noise at the edges of the spectrometers.
These edges correspond to the boundaries of the three spectrometers, which collectively cover the \gls{uv}, \gls{vio}, and \gls{vnir} light spectra.
The noisy edge ranges are as follows: 240.811-246.635 nm, 338.457-340.797 nm, 382.138-387.859 nm, 473.184-492.427 nm, and 849-905.574 nm.
In addition to being noisy regions, these regions do not contain any useful information related to each of the major oxides.
Consequently, these regions are masked by zeroing out the values, rather than removing them, as they represent meaningful variation in the data~\cite{cleggRecalibrationMarsScience2017}.

Additionally, as a result of the aforementioned preprocessing applied to the raw \gls{libs} data, negative values are present in the \gls{ccs} data.
These negative values are not physically meaningful, since you cannot have negative light intensity \cite{p9_paper}.
Similar to the noisy edges, these negative values are also masked by zeroing out the values.

We transpose the data so that each row represents a location and each column represents a wavelength feature. 
Each location is now represented as a vector of wavelengths, with the corresponding average intensity values for each wavelength. 
These vectors are then concatenated to form a tensor, giving us the full Averaged Intensity Tensor.

For each sample, we have a corresponding set of major oxide compositions in weight percentage (wt\%).
These compositions are used as the target labels for the machine learning models.
An excerpt of this data is shown in Table \ref{tab:composition_data_example}.
While the \textit{Target}, \textit{Spectrum Name}, and \textit{Sample Names} are part of the dataset, our analysis focuses primarily on the \textit{Sample Names}.
The concentrations of the eight oxides \ce{SiO2}, \ce{TiO2}, \ce{Al2O3}, \ce{FeO_T}, \ce{MnO}, \ce{MgO}, \ce{CaO}, \ce{Na2O}, and \ce{K2O} represent the expected values for these oxides in the sample, serving as our ground truth. The \textit{MOC total} is not utilized in this study.

\begin{table*}[h]
\centering
\begin{tabular}{lllllllllllll}
\toprule
     Target & Spectrum Name & Sample Name & \ce{SiO2} & \ce{TiO2} & \ce{Al2O3} & \ce{FeO_T} & \ce{MnO} & \ce{MgO} & \ce{CaO} & \ce{Na2O} & \ce{K2O} & \ce{MOC total} \\
\midrule
AGV2 & AGV2 & AGV2 & 59.3 & 1.05 & 16.91 & 6.02 & 0.099 & 1.79 & 5.2 & 4.19 & 2.88 & 97.44 \\
BCR-2 & BCR2 & BCR2 & 54.1 & 2.26 & 13.5 & 12.42 & 0.2 & 3.59 & 7.12 & 3.16 & 1.79 & 98.14 \\
$\vdots$ & $\vdots$ & $\vdots$ & $\vdots$ & $\vdots$ & $\vdots$ & $\vdots$ & $\vdots$ & $\vdots$ & $\vdots$ & $\vdots$ & $\vdots$ & $\vdots$ \\
TB & --- & --- & 60.23 & 0.93 & 20.64 & 11.6387 & 0.052 & 1.93 & 0.000031 & 1.32 & 3.87 & 100.610731 \\
    TB2 & --- & --- & 60.4 & 0.93 & 20.5 & 11.6536 & 0.047 & 1.86 & 0.2 & 1.29 & 3.86 & 100.7406 \\
\bottomrule
\end{tabular}
\caption{Excerpt from the composition dataset (from \citet{p9_paper})}
\label{tab:composition_data_example}
\end{table*}

The major oxide weight percentages are appended to the matrix of spectral data, forming the final dataset.
This dataset is shown in Table~\ref{tab:final_dataset_example}.
The \textit{Target} column corresponds to the sample name, while the \textit{ID} column contains the unique identifier for each location.

\begin{table*}[h]
\centering
\footnotesize
\begin{tabular}{llllllllllllllllllllll}
\toprule
    240.81   & $\cdots$     & 425.82    & 425.87   & $\cdots$ & 905.57  & \ce{SiO2} & \ce{TiO2} & \ce{Al2O3} & \ce{FeO_T} & \ce{MgO} & \ce{CaO} & \ce{Na2O} & \ce{K2O} & Target     & ID \\
\midrule
	0        & $\cdots$     & 1.53e+10 & 1.62e+10 & $\cdots$ & 0        & 56.13     & 0.69 & 17.69 & 5.86 & 3.85 & 7.07 & 3.32 & 1.44 & jsc1421     & jsc1421\_2013\_09\_12\_211002\_ccs \\
	0        & $\cdots$     & 1.28e+10 & 1.30e+10 & $\cdots$ & 0        & 56.13     & 0.69 & 17.69 & 5.86 & 3.85 & 7.07 & 3.32 & 1.44 & jsc1421     & jsc1421\_2013\_09\_12\_211143\_ccs \\
    0        & $\cdots$     & 1.87e+10 & 1.83e+10 & $\cdots$ & 0        & 56.13     & 0.69 & 17.69 & 5.86 & 3.85 & 7.07 & 3.32 & 1.44 & jsc1421     & jsc1421\_2013\_09\_12\_210628\_ccs \\
    0        & $\cdots$     & 1.77e+10 & 1.78e+10 & $\cdots$ & 0        & 56.13     & 0.69 & 17.69 & 5.86 & 3.85 & 7.07 & 3.32 & 1.44 & jsc1421     & jsc1421\_2013\_09\_12\_210415\_ccs \\
    0        & $\cdots$     & 1.75e+10 & 1.79e+10 & $\cdots$ & 0        & 56.13     & 0.69 & 17.69 & 5.86 & 3.85 & 7.07 & 3.32 & 1.44 & jsc1421     & jsc1421\_2013\_09\_12\_210811\_ccs \\
    0        & $\cdots$     & 5.52e+10 & 3.74e+10 & $\cdots$ & 0        & 57.60     & 0.78 & 26.60 & 2.73 & 0.70 & 0.01 & 0.38 & 7.10 & pg7         & pg7\_2013\_11\_07\_161903\_ccs \\
    0        & $\cdots$     & 5.09e+10 & 3.41e+10 & $\cdots$ & 0        & 57.60     & 0.78 & 26.60 & 2.73 & 0.70 & 0.01 & 0.38 & 7.10 & pg7         & pg7\_2013\_11\_07\_162038\_ccs \\
    0        & $\cdots$     & 5.99e+10 & 3.97e+10 & $\cdots$ & 0        & 57.60     & 0.78 & 26.60 & 2.73 & 0.70 & 0.01 & 0.38 & 7.10 & pg7         & pg7\_2013\_11\_07\_161422\_ccs \\
    0        & $\cdots$     & 5.22e+10 & 3.47e+10 & $\cdots$ & 0        & 57.60     & 0.78 & 26.60 & 2.73 & 0.70 & 0.01 & 0.38 & 7.10 & pg7         & pg7\_2013\_11\_07\_161735\_ccs \\
    0        & $\cdots$     & 5.29e+10 & 3.62e+10 & $\cdots$ & 0        & 57.60     & 0.78 & 26.60 & 2.73 & 0.70 & 0.01 & 0.38 & 7.10 & pg7         & pg7\_2013\_11\_07\_161552\_ccs \\
	$\vdots$ & $\cdots$ & $\vdots$ & $\vdots$ & $\cdots$ & $\vdots$ & $\vdots$ & $\vdots$ & $\vdots$ & $\vdots$ & $\vdots$ & $\vdots$ & $\vdots$ & $\vdots$ & $\vdots$ & $\vdots$ \\
\midrule
\end{tabular}
\caption{Excerpt from the final dataset (values have been rounded to two decimal places for brevity).}
\label{tab:final_dataset_example}
\end{table*}

\subsection{Experimental Setup}
Experiments were conducted on a machine equipped with an Intel Xeon Gold 6242 CPU, featuring 16 cores and 32 threads.
The CPU has a base clock speed of 2.80 GHz and a maximum turbo frequency of 3.90 GHz.
The system has 64 GB of RAM and runs on Ubuntu 22.04.2 LTS.
Models were implemented using Python 3.10.11.
The primary libraries used were Scikit-learn 1.4.2, XGBoost 2.0.3, Torch 2.2.2, NumPy 1.26.4, Pandas 2.2.1, Keras 3.2.1 and Optuna 3.6.1.

% \subsection{Validation and Testing Procedures}


\subsection{Summary}

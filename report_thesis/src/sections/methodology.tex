\section{Methodology}\label{sec:methodology}
In this section, we outline the methodology used in this study to address the challenges identified in the problem definition. Our objective was to identify the most promising machine learning models and preprocessing techniques proposed in the literature, outlined in Section~\ref{sec:related-work}, for predicting major oxide compositions from LIBS data. 
Then, using this knowledge, develop a pipeline that utilizes the strengths of these models and preprocessing techniques to improve prediction accuracy and robustness of the predictions.
We first present a description of the datasets used, how it was prepared and how it was split for model training. Then the preprocessing and model selection process is described followed by the experimental setup and the evaluation metrics employed. Finally, we describe our validation testing procedures and how the final model evaluations were conducted to ensure unbiased results.

\subsection{Data Prepraration}
Similarly to our previous work \cite{p9}, we used the publicly available \gls{ccs} data. 
CCS refers to LIBS data that has been through a series of preprocessing steps such as subtracting the ambient light background, noise removal and removing the electron continuum to derive data that is more suitable for quantitative analysis. 
The full description of this preprocessing can be found in \citet{wiensPreFlight3}.

While the CCS data is in a more suitable form for quantitative analysis, it still requires further preprocessing to be useful. Specifically, the data will still have noise at the boundaries of the spectrometer ranges, which have to be masked out. The data will also contain negative values, which are not physically meaningful and have to be set to zero.
For a more detailed description of the data and the preprocessing steps, we refer the reader to \citet{p9}, since the data preparation follows the same procedures detailed in that work.

Once preprocessed, the data was conventionally split into training and testing sets. 
However, as detailed in \citet{p9} and \cite{cleggRecalibrationMarsScience2017}, care has to be taken in order to ensure that the training set is constructed such that it has sufficient samples containing extreme values.
To incorporate these extreme values, the training set was constructed by including the two highest and two lowest extreme samples from each oxide category.
Following the inclusion of extreme samples, we adjusted the size of the test split to accommodate the reduced pool of normal samples. 
This adjustment was calculated by determining the number of extreme samples, subtracting this from the total sample count, and recalculating the test size proportionally. 
This ensures that the proportion of the test set remains appropriate relative to the now smaller dataset. 
Subsequently, we conducted the train-test split, where the normal samples --- those not categorized as extreme --- were divided according to the newly adjusted test size, with the extreme samples included to ensure exposure to varied data points.

\subsection{Model and Preprocessing Selection}
For the initial investigative experiments we selected a range of models for further exploration, namely Support Vector Regression (SVR), Gradient Boosting Regression (GBR), Partial Least Squares Regression (PLS), eXtreme Gradient Boosting (XGBoost), Natural Gradient Boosting (NGBoost), Extra Trees Regression (ETR), Elastic Net (ENet), and Stochastic Gradient Descent (SGD). 
We also included regular neural network (NN) and convolutional neural networks (CNN) in this phase of experimentation. 
This selection was guided primarily by the literature review but also by exploratory intuition,
aiming to discover potentially innovative applications and performances within our specific dataset.

Our literature review highlighted various approaches and their effectiveness in handling the challenges associated with predicting major oxide compositions from LIBS data. 
For instance, Anderson et al. discussed the use of multiple regression models, finding that different models excelled with specific oxides, which informed our model-specific approach. 
Song et al. presented a hybrid model combining domain knowledge with machine learning, which inspired our interest in models that could offer both high performance and interpretability. Rezaei et al. demonstrated the beneficial impact of dimensionality reduction techniques like PCA, which we considered essential for managing our high-dimensional LIBS data.

All models demonstrated robust performance; however, \gls{gbr}, \gls{svr}, \gls{xgboost}, \gls{etr}, and \gls{pls} consistently excelled across all oxides. 
Notably, we observed model-specific strengths for certain oxides—SVR was particularly effective for \ce{SiO2}, while \gls{gbr} excelled with \ce{FeO_T}. 
This differential performance prompted us to explore architectural frameworks that could systematically capitalize on the strengths of each model for specific oxides.

A particularly exciting discovery from our literature review was a study on the stacking and chaining of normalization methods initially aimed at classification contexts. 
Inspired by these ideas, we explored the possibility of improving model performance by optimizing the preprocessing chain for each model, per oxide, in order to determine which normalization methods would be most beneficial.
For this purpose, we considered the following preprocessing techniques: Min-Max Scaling, Standard Scaling, Robust Scaling, MaxAbs Scaling, Quantile Transformer and Power Transformer. 
Additionally, we also considered dimensionality reduction techniques such as PCA and KernelPCA.
Through experimentation, we identified which preprocessing techniques were most effective for each model and oxide. 

Realizing the potential of this, we decided to further investigate how the predictive accuracy and robustness could be improved by combining the strengths of these models and preprocessing techniques.
From our investigation, we identified that the stacking ensemble method would be the most suitable and novel approach for our problem.
Consequently, we concluded that a stacking ensemble method would optimize outcomes for our dataset. 

Stacking ensemble is analogous to the methodologies employed in the original MOC pipeline, which also tailored predictions for each oxide by blending outputs from the PLS-SM and ICA phases, variably weighting the influence of the ICA predictions depending on the oxide.
Unlike the MOC model, which required manual determination of the model weightings for each oxide, our method utilizes a meta learner to learn optimal parameter settings. 
Stacking ensemble is beneficial as it dynamically adapts to our dataset's characteristics without the need for domain-specific knowledge.
This approach represents a more sophisticated method, streamlining complex model configurations and potentially enhancing predictive accuracy through dynamically learned integrations, rather than fixed presets.

\subsection{Experimental Setup}

\subsection{Evaluation Metrics}
To evaluate the performance of our models in predicting major oxide compositions from \gls{libs} data, we will use two key metrics: \gls{rmse} and standard deviation of prediction errors.

\gls{rmse} will be used as a measure of accuracy, quantifying the difference between the predicted and actual values of the major oxides in the samples. It is defined by the equation:

\begin{equation}
    RMSE = \sqrt{\frac{1}{n} \sum_{i=1}^{n} (y_i - \hat{y}_i)^2},
\end{equation}

where $y_i$ represents the actual values, $\hat{y}_i$ the predicted values, and $n$ the number of observations. A lower RMSE indicates better accuracy.

To assess the robustness of our models, we will consider the standard deviation of prediction errors across each oxide and test instance. This metric measures the variability of the prediction errors and provides insight into the consistency of the model's performance. It is defined as:

\begin{equation}
    \sigma_{error} = \sqrt{\frac{1}{n-1} \sum_{i=1}^{n} (e_i - \bar{e})^2},
\end{equation}

where $e_i = y_i - \hat{y}_i$ and $\bar{e}$ is the mean error. A lower standard deviation indicates better robustness.

By using these two metrics, we aim to evaluate model performance in terms of both accuracy and robustness, which are crucial for the reliable prediction of major oxide compositions from \gls{libs} data.

\subsection{Validation and Testing Procedures}


\subsubsection{Summary}

\section{Problem Definition}\label{sec:problem_definition}

\subsection{Motivating Example: NASA's Mars Missions}
The NASA Viking missions in the 1970s were the first to successfully land on Mars, aiming to determine if life existed on the planet. 
While these missions advanced our knowledge of the Martian environment, the search for evidence of life remained inconclusive~\cite{marsnasagov_vikings}.

Subsequent missions, such as the \gls{mer} mission in 2003 and the \gls{msl} mission in 2012, sought to investigate whether Mars ever had the conditions to support life. 
The Curiosity rover, part of the \gls{msl} mission, is equipped with the \gls{chemcam} instrument, which uses \gls{libs} to gather spectral data from geological samples on Mars~\cite{wiensChemcam2012}.

\gls{libs} uses a laser to ablate surface material and generate a plasma plume, which emits light captured by spectrometers. 
The resulting spectra consist of emission lines associated with the concentration of specific elements, serving as a multi-dimensional fingerprint of the sample's elemental composition~\cite{cleggRecalibrationMarsScience2017}.

\subsection{Problem Formulation}
Predicting major oxide compositions from \gls{libs} data presents significant computational challenges, including high dimensionality, non-linearity, multicollinearity, and the phenomenon known as matrix effects~\cite{andersonImprovedAccuracyQuantitative2017}.
Furthermore, the high cost of data collection often results in small datasets, complicating the task of building accurate and robust models.

Our problem definition builds upon the definitions provided in \citet{p9_paper}.

% TODO: Write data definitions here; define problem formally.

\textbf{Problem Definition:} This thesis aims to address the challenges in predicting major oxide compositions from \gls{libs} data by developing machine learning models that improve the accuracy and robustness of these predictions. 
We will investigate various techniques to handle the high dimensionality, non-linearity, and small dataset size inherent in this problem, and evaluate model performance using appropriate metrics, which will be discussed in detail in Section~\ref{sec:methodology}.

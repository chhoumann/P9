\section{Problem Definition}\label{sec:problem_definition}
There remains considerable uncertainty about which machine learning techniques best predict the composition of major oxides in Martian geological samples using \gls{libs} data.
In addition, predicting major oxide compositions from \gls{libs} data presents significant computational challenges, including the high dimensionality and non-linearity of the data, compounded by multicollinearity and the phenomenon known as matrix effects~\cite{andersonImprovedAccuracyQuantitative2017}.
These effects can cause the intensity of emission lines from an element to vary independently of that element's concentration, introducing unknown variables that complicate the analysis.
Furthermore, due to the high cost of data collection, datasets are often small, which further complicates the task of building accurate and robust models.

Building upon the baseline established in~\citet{p9_paper}, our work aims to address the significant challenges inherent in predicting major oxide compositions from \gls{libs} data by improving the accuracy and robustness of these predictions.
Here, we define accuracy as the ability of a model to predict the composition of major oxides in Martian geological samples, while robustness refers to the stability of these predictions across different samples and oxides.

% Placeholder
\textit{Problem Definition:} This thesis aims to address the challenges in predicting major oxide compositions from \gls{libs} data by developing machine learning models that improve the accuracy and robustness of these predictions. 

We will investigate various techniques to handle the high dimensionality, non-linearity, and small dataset size inherent in this problem, and evaluate the performance of these models using appropriate metrics, which will be discussed in detail in Section~\ref{sec:methodology}.

\section{Problem Definition}\label{sec:problem_definition}
The primary objective of this research is to enhance computational methods for the accurate and robust quantification of chemical compositions using \gls{libs} data.
This enhancement seeks to improve the adaptability and precision of \gls{libs} across various environmental contexts, including terrestrial and planetary geological samples.

Quantifying chemical compositions from \gls{libs} spectral data poses significant challenges due to the high dimensionality of the data and the complex physical and chemical interactions involved. The interpretation of \gls{libs} data is particularly complicated by factors such as multicollinearity among spectral features and matrix effects, which can obscure the relationships between the observed spectra and the actual chemical compositions.

The process of elemental quantification using \gls{libs} data is constrained by several factors:
\begin{itemize}
    \item \textbf{Data Dimensionality and Collinearity:} High dimensionality of spectral data coupled with multicollinearity, where multiple spectral features may exhibit strong correlations, complicates the modeling and analysis\cite{andersonImprovedAccuracyQuantitative2017}.
    \item \textbf{Matrix Effects:} Used as a catch-all term that encompasses any effect that can cause the intensity of emission lines from an element to vary independent of that element's concentration. Different background materials can alter the emission intensities, pose significant challenges in accurately quantifying elemental concentrations. The spectra are complex due to the interaction of multiple physical processes including the coupling process between the laser photons and the target, self-absorption of optical emission lines within the plasma, recombination of elements into molecules, and collisional interactions in the plasma\cite{cleggRecalibrationMarsScience2017, andersonImprovedAccuracyQuantitative2017}.
    \item \textbf{Data Availability:} Due to the high cost of data collection, datasets are often small, which may limit the generalizability of the models\cite{p9_paper}.
\end{itemize}

These constraints necessitate the development of advanced computational models that can effectively account for and mitigate these complexities to improve the accuracy and reliability of chemical composition analysis using \gls{libs} data.

The input to our computational models consists of \gls{libs} spectral data, which includes intensity readings across a spectrum of wavelengths. This data is in the form of Clean, Calibrated Spectra\cite{andersonImprovedAccuracyQuantitative2017}, the output of level 1 processing as described by \citet{wiensPreflightCalibrationInitial2013}. The wavelength intensities are quantified in units of photon/shot/mm\textsuperscript{2}/sr/nm.

We have:

\begin{itemize}
    \item \textbf{Matrix $A_{t \times o}$}: This matrix denotes the chemical concentrations in weight percent for oxides $o$ across targets $t$.
    \item \textbf{Matrix $B_{w \times s}$}: A Boolean matrix that links wavelengths $w$ to spectrometers $s$, indicating whether a specific wavelength is detected by a spectrometer.
    \item \textbf{Matrix $C_{t \times l \times s \times w}$}: Holds the spectral intensity data, where each entry represents the intensity recorded for a target $t$ at location $l$, for shot $s$, at wavelength $w$.
    \item \textbf{Matrix $D_{t \times l \times w}$}: Derived from matrix $C$ by averaging the intensities across shots to provide a clearer signal for each location and wavelength:
    \[
    D[t, l, w] = \frac{1}{|S|} \sum_{s \in S} C[t, l, s, w].
    \]
    \item \textbf{Matrix $E_{t \times l \times w}$}: Is the result of $D$ processed by applying wavelength-specific masks, setting intensities to zero in masking ranges to focus on relevant spectral features.
\end{itemize}

The outputs of our models are the quantified chemical compositions of geological samples. These are primarily the concentrations of major elements, represented in weight percentage. While trace elements are also present, our current analysis does not quantify these.
Our goal is to construct a mapping function $\mathcal{F} : \mathbb{R}^N \rightarrow \mathbb{R}^O$, where $N$ represents the dimensionality of the processed LIBS signals, and $O$ represents the number of target oxides. This function maps a processed LIBS signal vector $\mathbf{x} \in \mathbb{R}^N$ to a vector $\mathbf{v} \in \mathbb{R}^O$ of estimated oxide concentrations:
\[
\mathbf{v} = \mathcal{F}(\mathbf{x}).
\]

The primary metric used to evaluate the performance of our computational models is the \gls{rmse}. \gls{rmse} is calculated using the formula:
\[
\text{RMSE} = \sqrt{\frac{1}{n} \sum_{i=1}^{n} (\mathbf{v}_i - \hat{\mathbf{v}}_i)^2}
\]
where \( \mathbf{v}_i \) is the vector of actual oxide concentrations for the \( i \)-th sample, \( \hat{\mathbf{v}}_i \) is the corresponding vector of predicted oxide concentrations, and \( n \) is the total number of samples. This measure quantifies the average magnitude of the prediction error across all predicted values, providing a clear indication of model accuracy in quantifying chemical compositions.

Our evaluation involves a combination of methodologies. Initially, the model is validated using a train/test split, where it is trained on a subset of data and tested on a separate subset to assess its predictive performance. Further, we compare the performance of our models against a baseline model previously established in our research \cite{p9_paper}, as well as against chemical analysis methods and other computational models documented in existing literature. This comprehensive comparison helps to underscore the improvements and benefits our approach offers over traditional methods and prior computational approaches.

\textbf{Problem Definition:} This thesis aims to address the challenges in predicting major oxide compositions from \gls{libs} data by enhancing computational methods to improve accuracy and robustness. We propose to develop computational models capable of effectively accounting for and mitigating the complexities inherent in \gls{libs} data. Our models will take as input a matrix in the form of $E$, as well as ground truth data in the form of $A$, to construct a mapping function $\mathcal{F} : \mathbb{R}^N \rightarrow \mathbb{R}^O$, mapping processed LIBS signals to estimated oxide concentrations. Success will be evaluated primarily through the \gls{rmse} metric, comparing the predictive accuracy of our models against existing benchmarks and baseline models established in prior research.


\subsection{Motivating Example: NASA's Mars Missions}
The NASA Viking missions in the 1970s were the first to successfully land on Mars, aiming to determine if life existed on the planet. 
While these missions advanced our knowledge of the Martian environment, the search for evidence of life remained inconclusive~\cite{marsnasagov_vikings}.

Subsequent missions, such as the \gls{mer} mission in 2003 and the \gls{msl} mission in 2012, sought to investigate whether Mars ever had the conditions to support life. 
The Curiosity rover, part of the \gls{msl} mission, is equipped with the \gls{chemcam} instrument, which uses \gls{libs} to gather spectral data from geological samples on Mars~\cite{wiensChemcam2012}.

\gls{libs} uses a laser to ablate surface material and generate a plasma plume, which emits light captured by spectrometers. 
The resulting spectra consist of emission lines associated with the concentration of specific elements, serving as a multi-dimensional fingerprint of the sample's elemental composition~\cite{cleggRecalibrationMarsScience2017}.

\subsection{Motivating Example: NASA's Mars Missions}

The quest for understanding Mars' geological and potential biological history has been significantly advanced by NASA's Mars missions, particularly through the deployment of sophisticated instrumentation such as the \gls{chemcam} on the Curiosity rover. This instrument, which employs \gls{libs}, has been pivotal in analyzing Martian soil and rock samples to determine their chemical compositions. The \gls{libs} technique involves ablating surface material with a laser to generate a plasma plume whose emitted light is captured and analyzed to identify the elemental composition of the sample\cite{cleggRecalibrationMarsScience2017}.

\textbf{Sample Input:} The input for our computational models includes LIBS spectral data captured by the Curiosity rover. This data comprises intensity readings across various wavelengths, representing different mineral compositions in Martian soil samples, such as iron, magnesium, and silicon oxides.

\textbf{Sample Output:} The output from our models includes detailed quantifications of these oxides in weight percentages, providing insights into the geological history and current state of Martian soil.

\textbf{Explanation:} The process involves several computational steps: initially, the spectral data undergoes preprocessing to clean and calibrate it, as detailed in \citet{wiensChemcam2012}. Post-preprocessing, we employ machine learning models, specifically a recreated version of the Multivariate Oxide Composition (MOC) pipeline \cite{cleggRecalibrationMarsScience2017}, to predict the composition of major oxides. This pipeline, enhanced in our study \cite{p9_paper}, includes steps such as feature extraction and the application of regression techniques, aiming to overcome challenges like data multicollinearity and matrix effects, thus improving the predictive accuracy of elemental quantifications.

This example underscores the practical application of our computational advancements in a real-world setting, providing a clear connection between theoretical development and practical implementation in planetary science.




\subsection{Motivating Example: NASA's Mars Missions}

NASA's exploration of Mars, beginning with the Viking missions in the 1970s, has progressively deepened our understanding of Mars. The Mars Science Laboratory (MSL) mission, which landed the Curiosity rover in Gale Crater in 2012, represents a pivotal step in this journey. Curiosity is equipped with the \gls{chemcam} instrument, a tool that uses \gls{libs} to analyze the chemical composition of Martian rocks and soils directly and non-invasively.
\gls{libs} is uniquely suited for the harsh Martian environment due to its ability to remotely zap rocks and soils up to 7 meters away, creating a plasma that can be spectrally analyzed to determine the elemental composition of the vaporized material. This capability is crucial because it allows scientists to quickly and efficiently assess the geochemistry of multiple sites without physically moving the rover, thus conservatively managing the rover's limited energy and resources. The mission's focus has been on assessing past habitability, and the data gathered by \gls{chemcam} has been instrumental in identifying environments that could have supported life.

Computationally, the task of quantifying the oxides in Martian rock and soil samples begins with the \gls{libs} spectral data collected by Curiosity. This data comprises high-dimensional spectra with thousands of potential features, each corresponding to a specific element's emission lines. The computational challenge lies in accurately interpreting these complex data sets to deduce the concentrations of various elements, especially major oxides like iron, magnesium, and silicon, which are crucial for understanding Martian geology.

Computational models take these spectral readings as input and process them through several stages of analysis. Initially, the data undergoes preprocessing to correct for any instrumental effects and to calibrate the raw spectra. This step ensures that the readings are accurate and can be reliably used for quantitative analysis. The cleaned data is then fed into machine learning models, which have been trained to correlate specific spectral signatures with known concentrations of elements, based on a database of Earth-based and synthetic Martian analogues.

The output of this process is a detailed quantitative analysis of the soil or rock sample's chemical composition, expressed in weight percentages of each detected oxide. This output not only informs about the current state of Martian geology but also provides insights into the planet's past environmental conditions.

The use of \gls{libs} on the Curiosity rover within the MSL mission shows how computational advancements can enhance our understanding of extraterrestrial geology. By effectively quantifying chemical compositions from \gls{libs} data, we can infer the historical climatic conditions of Mars, offering clues to its past habitability. This example vividly demonstrates the synergy between technological innovation and computational science, driving forward our exploration and understanding of Mars.

\subsubsection{Evaluation Metrics}\label{subsubsec:evaluation_metrics}
As mentioned in Section~\ref{sec:problem_definition}, the performance of the models was quantitatively assessed using the \gls{rmse} and the standard deviation of the residuals.
These metrics are calculated for each fold and averaged across all folds to provide comprehensive indicators of model accuracy and variability.
In addition, we also compute the metrics for the test set to provide a measure of the model's performance on unseen data.
Therefore, we have the following metrics for each experiment:
\begin{enumerate}
    \item \textbf{Fold-specific \gls{rmse} and Standard Deviation:} For each of the $k$ folds, we calculate both the \gls{rmse} and standard deviation, denoted as \texttt{rmse\_cv\_n} and \texttt{std\_dev\_cv\_n}, where \texttt{n} ranges from 1 to $k$.
    \item \textbf{Average \gls{rmse} and Standard Deviation:} The overall cross-validation \gls{rmse} (\texttt{rmse\_cv}) and standard deviation (\texttt{std\_dev\_cv}) are computed as the mean of the fold-specific values.
    \item \textbf{Test Set \gls{rmse} and Standard Deviation:} The \gls{rmse} and standard deviation are also computed for the test set, denoted as \texttt{rmsep} and \texttt{std\_dev}, to provide a measure of the model's performance on unseen data.
\end{enumerate}

K-fold cross-validation provides a robust estimate of model performance by averaging metrics over multiple folds, reducing variance and offering a clearer picture of the model's generalizability.
Evaluating the model on a separate test set representative of unseen data ensures that performance metrics accurately reflect the model's generalization capability.
However, our data partitioning method, which moves the most extreme values into the training data, naturally results in the testing data being closer to the mean of the data distribution, making it easier to predict.
In practice, this would result in lower \texttt{rmsep} and \texttt{std\_dev} values compared to the cross validation metrics.
Therefore, evaluating the model's performance using both the cross-validation metrics (\texttt{rmse\_cv} and \texttt{std\_dev\_cv}) and the test set metrics (\texttt{rmsep} and \texttt{std\_dev}) is crucial.
The cross-validation metrics provide insights into the model's stability across different subsets, while the test set metrics offer a final measure of performance on truly unseen data, giving a comprehensive assessment of the model's generalizability.

To ensure generalizability for our models, we will employ a k-fold cross-validation strategy, for partitioning the data into training and test sets.
However, as stated in Section~\ref{sec:problem_definition} the Masked Intensity Tensor contains multiple locations for each sample.
This necessitates careful considerations when constructing each fold to prevent data leakage of location data across the training and test partitions.
To this end, we have devised a custom k-fold cross validation method to address this issue.
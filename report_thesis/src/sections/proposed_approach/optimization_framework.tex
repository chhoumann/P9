\subsection{Optimization Framework}\label{sec:optimization_framework}
One of the primary challenges in developing a stacking ensemble is determining the optimal choice of base estimators. \citet{wolpertstacked_1992} highlighted that this can be considered a 'black art' and that the choice usually relies on intelligent guesses. 
In our case, this problem is further exacerbated by the fact that the optimal choice of base estimator may vary depending on the target oxide. 
The complexity of the problem is increased because different oxides require different models, and the optimal preprocessing techniques will depend on both the model and the specific oxide being predicted. 
Additionally, each estimator may require different variations of preprocessing depending on the oxide. 
Due to challenges such as matrix effects, multicollinearity, and high dimensionality caused by very complex physical processes, it is difficult to determine which configuration is optimal.
Selecting the appropriate preprocessing steps for each base estimator is essential, as incorrect preprocessing can significantly degrade performance and undermine the model's effectiveness
Furthermore, choosing the right hyperparameters for each base estimator introduces additional complexity, as these decisions also significantly impact model performance and must be carefully tuned for each specific oxide. 
Some estimators might require very little tuning to achieve accurate and robust predictions, while others might require extensive tuning, depending on the target oxide. 
For instance, simpler models like Elastic Net and Ridge Regression might quickly yield satisfactory results with minimal hyperparameter adjustments. 
In contrast, more complex models like Convolutional Neural Networks or Gradient Boosting Regression involve a larger number of hyperparameters that need fine-tuning to perform well. 
The extent of tuning required is also influenced by the characteristics of the target oxide, such as its data distribution, noise levels, and feature interactions. 
These factors can affect how sensitive an estimator is to its hyperparameters.
Finally, hyperparameters cannot be considered in isolation, because depending on the preprocessing steps applied to the data, the optimal hyperparameters may vary.
Given these complexities, we need a systematic approach to determine the optimal configuration of hyperparameters and preprocessing steps tailored to each estimator and oxide.

To guide this process we have developed a working assumption.
Namely, that we assume that selecting the top-$n$ best pipelines for each oxide, given different preprocessors and models per pipeline, will yield the best pipelines for a given oxide in our stacking ensemble.
Here, $n$ is a heuristic based on the results and \textit{best} is evaluated in terms of the metrics outlined in Section~\ref{subsec:evaluation_metrics}.
Additionaly, each permutation will utilize our proposed data partitioning and cross-validation strategy outlined in Section~\ref{subsec:validation_testing_procedures}.
Utilizing our proposed data partitioning and cross-validation strategy, along with the aformentioned evaluation metrics, will ensure that the top-$n$ pipelines align with our goals of generalization, robustness, and accuracy outlined in Section~\ref{sec:problem_definition}.
This reduces the problem to finding a selection of preprocessors and models, and a means for doing a guided search that evaluates multiple permutations of these to identify the top-$n$ best pipeline for each oxide.

To address the first part of this challenge, we selected a diverse set of models and preprocessing techniques, as outlined in Section~\ref{sec:model-selection}.
In order to address the second part of the problem, we have developed an optimization framework, which will be described in the following section.

\subsubsection{The Framework}
% Introduce the fact that we needed a tool that fulfills several criteria
    % Easy to use
    % Lots of possible configurations
    % Efficient search strategy
    % Ability to balance exploration and exploitation
    % Configurable
    % Integrate well with your exisiting code (data partitioning and cross-validation strategy)

To develop a framework to address this challenge meant that several criteria had to be fulfilled.
Namely, the framework had to be able to search through a combined search space of models and preprocessing techniques.
It should also be able to integrate with our existing tools, such as the data partitioning and cross-validation strategy.
The framework
Finally, time and computational resources should be managed efficiently, while still arriving at well performing configurations.
The first and second criteria could be fulfilled by most hyperparameter optimization methods,



The framework had to be able to search through a combined search space of models and preprocessing techniques.
This could be done by traditional methods such as grid search or random search, but due to the vast search space, these methods would require an infeasible amount of time to correctly identify the optimal pipelines for each oxide.
This meant that the framework also needed to have an efficient search strategy.
Ideally, the framework should be able to balance exploration and exploitation of the search space.
This would allow the framework to identify multiple promising configurations, but also focus on the most promising.
Additionally, not all configurations will be promising. 
Some permutations of hyperparameters, combination of models and preprocessing techniques or both, may not be optimal.
This could to scenarios where the framework is wasting computational resources on configurations that should be terminated early.
Therefore, the framework should have a mechanism to terminate unpromising trials early.
Finally, the framework should be able to optimize for multiple metrics simultaneously, because as outlined in Section~\ref{subsec:validation_testing_procedures} evaluating solely on the RMSEP may lead to misleading and poor results.
For these reasons, we chose to use Optuna as the basis of our optimization framework.

Optuna is a hyperparameter optimization framework designed for direct integration with Python.
The framework's flexible search space definition allows it to dynamically adjust and optimize hyperparameters during execution. 
This flexibility enables the exploration of a wide range of hyperparameter configurations, adapting to the specific needs and complexities of different modeling scenarios. 
By supporting conditional and dependent hyperparameters, the framework can handle more sophisticated models and optimization strategies, ensuring that the best possible configurations are identified even in highly complex spaces.
Additionally, Optuna uses advanced sampling strategies to explore promising areas of the search space and employs pruning techniques to terminate unpromising trials early, optimizing computational resource use. These combined features enable efficient and effective hyperparameter optimization, balancing exploration and exploitation while conserving computational resources. \cite{optuna_2019}
However, it should be noted that while Optuna is an ideal choice for our optimization framework, it is not without its limitations.
Because there is no one-size-fits-all solution for selecting the search strategy, how frequently to explore versus exploiting the search space and which pruning strategy to use there will still be choices that are heuristic in nature.
Nevertheless, Optuna provides solid options for all of these and the framework makes adjusting these choices easy.
Therefore, using Optuna as the basis for our optimization framework, combining it with our data partitioning and cross-validation strategy, we have a systematic approach to identifying the top-$n$ best pipelines for each oxide.

The complete optimization framework is outlined in Algorithm~\ref{alg:hyperparameter_optimization_framework}.

\begin{algorithm}
\caption{Hyperparameter Optimization Framework}
\label{alg:hyperparameter_optimization_framework}
\begin{algorithmic}[1]
\Require Dataset $D$, Model $m$, Number of Trials $N$, Random Seed $seed$, Sampler \texttt{sampler}, Pruner \texttt{pruner}
\Ensure Trial data, including metrics and configuration for each trial

\State \textbf{Initialize:} Set random seed for reproducibility if seed is not None \label{step:initialize}
\For{each trial $t$ from 1 to $N$} \label{step:trial_loop}
    \State $hp \gets \texttt{sample\_hyperparameters}(m, \texttt{sampler})$ \label{step:sample_hyperparameters}
    \State $m' \gets \texttt{instantiate\_model}(m, hp)$ \label{step:instantiate_model}
    \Statex
    \State $s\_params \gets \texttt{sample\_scaler\_params}(\texttt{sampler})$ \label{step:sample_scaler_params}
    \State $s \gets \texttt{instantiate\_scaler}(s\_params)$ \label{step:instantiate_scaler}
    \State $t\_params \gets \texttt{sample\_transformer\_params}(\texttt{sampler})$ \label{step:sample_transformer_params}
    \State $t \gets \texttt{instantiate\_transformer}(t\_params)$ \newline \hspace*{3em} \textbf{or} NONE \label{step:instantiate_transformer}
    \State $dr\_params \gets \texttt{sample\_dim\_reduction\_params}(\texttt{sampler})$ \label{step:sample_dim_reduction_params}
    \State $dr \gets \texttt{instantiate\_dim\_reduction}(dr\_params)$ \newline \hspace*{3em}\textbf{or} NONE \label{step:instantiate_dim_reduction}
    \Statex     
    \State $pipeline \gets [s, t, dr]$ \textbf{or} $[s, NONE, NONE]$ \label{step:construct_pipeline}
    \State $T_{cv}, D_{train}, D_{test} \gets \text{apply data partitioning to } D$ \label{step:data_partitioning}
    \Statex
    \State $D_{train}'$ $\gets$ \texttt{pipeline\_apply}($D_{train}$, \texttt{pipeline}) \label{step:apply_pipeline_train}
    \State $T_{cv}'$ $\gets$ \texttt{pipeline\_apply}($T_{cv}$, \texttt{pipeline}) \label{step:apply_pipeline_cv}
    \Statex
    \State $CV_{metrics} \gets \texttt{cross\_validate}(m', T_{cv}')$ \label{step:cross_validate}
    \State $rmse_{cv} \gets \texttt{average}(CV_{metrics}.\texttt{rmse\_values})$ \label{step:average_rmse_cv}
    \State $std\_dev_{cv} \gets \texttt{average}(CV_{metrics}.\texttt{std\_dev\_values})$ \label{step:average_std_dev_cv}
    \Statex
    \State $m'_{train}$ $\gets$ \texttt{train}($m'$, $D_{train}'$) \label{step:train_model}
    \State $rmsep$, $\sigma_{error, test}$ $\gets$ \texttt{evaluate}($m'_{train}$, $D_{test}$) \label{step:evaluate_model}
    \Statex
    % function that stores the metrics
    \State \texttt{store\_metrics}($t$, $m'$, $pipeline$, $rmse\_cv$, \newline \hspace*{8em}$std\_dev\_cv$, $rmsep$, $\sigma_{error, test}$) \label{step:store_metrics}
\EndFor
\State \Return \texttt{rmse\_cv}, \texttt{std\_dev\_cv}, \texttt{rmsep}, $\sigma_{error, test}$ \label{step:return_metrics}
\end{algorithmic}
\end{algorithm}

Before describing the algorithm, we will first explain the details that are not covered in the algorithm.
Two central constructs of the Optuna framework is the \texttt{Study} object and the objective function.
This \texttt{Study} object is responsible for orchestrating the optimization process and is responsible for instantiating the search strategy, the pruning strategy, and the objective for the objective function, which are hyperparameters.
Because we are trying to identify the top-$n$ best pipelines on a per-oxide basis and since we are optimizing with one model at a time, we will have one \texttt{Study} object per model.
This means that each oxide will be optimizied for, with one model at a time, but with many permutations of preprocessors and hyperparameters. 
The objective function contains the logic for our optimization process and has a hyperparameter that controls the number of trials to be run.
It is also the objective which we are trying to minimize, which in our case is the \texttt{rmse\_cv} and \texttt{std\_dev\_cv}.

The algorithm starts by initializing the random seed for reproducibility (line~\ref{step:initialize}).
For each trial, the algorithm samples hyperparameters for the model (line~\ref{step:sample_hyperparameters}) and instantiates the model with the sampled hyperparameters (line~\ref{step:instantiate_model}).
Next, it samples hyperparameters for the scaler (line~\ref{step:sample_scaler_params}), transformer (line~\ref{step:sample_transformer_params}), and dimensionality reduction (line~\ref{step:sample_dim_reduction_params}) techniques and instantiates them (lines~\ref{step:instantiate_scaler} to \ref{step:instantiate_dim_reduction}).
The algorithm constructs the preprocessor pipeline in the order that they are defined (line~\ref{step:construct_pipeline}).
The data is partitioned using our partitioning strategy, yielding the cross-validation folds, as well as, the training and test set (line~\ref{step:data_partitioning}).
Recall that, to ensure that we are able to evaluate the models generalizability and accuracy on unseen data, we have to generate both the cross-validation folds and the training and test set.
It then applies the pipeline to the training and cross-validation data (lines~\ref{step:apply_pipeline_train} and~\ref{step:apply_pipeline_cv}).
Next, it cross-validates the model (line~\ref{step:cross_validate}) and calculates the average RMSE and standard deviation of the RMSE (lines~\ref{step:average_rmse_cv} and~\ref{step:average_std_dev_cv}).
The model is then trained on the training data (line~\ref{step:train_model}) and evaluated on the test data (line~\ref{step:evaluate_model}).
Finally, the metrics for the trial are stored (line~\ref{step:store_metrics}) and the metrics for all trials are returned (line~\ref{step:return_metrics}).
\subsection{Optimization Framework}\label{sec:optimization-framework}
One of the primary challenges in developing a stacking ensemble is determining the optimal choice of base estimators.
\citet{wolpert1992stacked} highlighted that this can be considered a 'black art' and that the choice usually relies on intelligent guesses.
In our case, this problem is further exacerbated by the fact that the optimal choice of base estimator may vary depending on the target oxide and the fact that we are also considering the optimal preprocessing techniques for each base estimator, as well as, the hyperparameters for each.
This means, that we are considering an entire pipeline of preprocessing techniques and base estimators for each target oxide.
Therefore, we need a systematic approach to determine the optimal configuration for each pipeline.
To guide this process we have developed a working assumption.
Namely, that we assume that selecting the top-$n$ best pipelines for each oxide, given different preprocessors and models per pipeline, will yield the best pipelines for a given oxide in our stacking ensemble.
Here, $n$ is a heuristic based on the results and \textit{best} is evaluated in terms of the metrics outlined in Section~\ref{sec:evaluation_metrics}.
Additionaly, each permutation will utilize our proposed data partitioning and cross-validation strategy outlined in Section~\ref{subsec:validation_testing_procedures}.
Utilizing our proposed data partitioning and cross-validation strategy, along with the aformentioned evaluation metrics, will ensure that the top-$n$ pipelines align with our goals of generalization, robustness, and accuracy outlined in Section~\ref{sec:problem-definition}.
This reduces the problem to finding a selection of preprocessors and models, and a method of testing multiple permutations of these to identify the top-$n$ best pipeline for each oxide.
In Section~\ref{sec:model-selection} we outlined the models and preprocessing techniques that we intend to use and in the following section we will describe the optimization framework that we have developed to address this challenge.

\subsubsection{The Framework}
The framework developed to address this challenge was constructed based on a specific set of criteria.
The framework had to be able to search through a combined search space of models and preprocessing techniques.
This could be done by traditional methods such as grid search or random search, but due to the vast search space, these methods would require an infeasible amount of time to correctly identify the optimal pipelines for each oxide.
This meant that the framework also needed to have an efficient search strategy.
Ideally, the framework should be able to balance exploration and exploitation of the search space.
This would allow the framework to identify multiple promising configurations, but also focus on the most promising.
Additionally, not all configurations will be promising. 
Some permutations of hyperparameters, combination of models and preprocessing techniques or both, may not be optimal.
This could to scenarios where the framework is wasting computational resources on configurations that should be terminated early.
Therefore, the framework should have a mechanism to terminate unpromising trials early.
Finally, the framework should be able to optimize for multiple metrics simultaneously, because as outlined in Section~\ref{subsec:validation_testing_procedures} evaluating solely on the RMSEP may lead to misleading and poor results.
For these reasons, we chose to use Optuna as the basis of our optimization framework.

Optuna is a hyperparameter optimization framework designed for direct integration with Python.
The framework's flexible search space definition allows it to dynamically adjust and optimize hyperparameters during execution. 
This flexibility enables the exploration of a wide range of hyperparameter configurations, adapting to the specific needs and complexities of different modeling scenarios. 
By supporting conditional and dependent hyperparameters, the framework can handle more sophisticated models and optimization strategies, ensuring that the best possible configurations are identified even in highly complex spaces.
Additionally, Optuna uses advanced sampling strategies to explore promising areas of the search space and employs pruning techniques to terminate unpromising trials early, optimizing computational resource use. These combined features enable efficient and effective hyperparameter optimization, balancing exploration and exploitation while conserving computational resources. \cite{optuna_2019}
However, it should be noted that while Optuna is an ideal choice for our optimization framework, it is not without its limitations.
Because there is no one-size-fits-all solution for selecting the search strategy, how frequently to explore versus exploiting the search space and which pruning strategy to use there will still be choices that are heuristic in nature.
Nevertheless, Optuna provides solid options for all of these and the framework makes adjusting these choices easy.
Therefore, using Optuna as the basis for our optimization framework, combining it with our data partitioning and cross-validation strategy, we have a systematic approach to identifying the top-$n$ best pipelines for each oxide.

\begin{algorithm}
\caption{Hyperparameter Optimization Framework}
\label{alg:hyperparameter_optimization_framework}
\begin{algorithmic}[1]
\Require Dataset $D$, Model $m$, Number of Trials $N$, Random Seed $seed$, Sampler \texttt{sampler}, Pruner \texttt{pruner}
\Ensure Trial data, including metrics and configuration for each trial

\State \textbf{Initialize:} Set random seed for reproducibility if seed is not None \label{step:initialize}
\For{each trial $t$ from 1 to $N$} \label{step:trial_loop}
    \State $hp \gets \texttt{sample\_hyperparameters}(m, \texttt{sampler})$ \label{step:sample_hyperparameters}
    \State $m' \gets \texttt{instantiate\_model}(m, hp)$ \label{step:instantiate_model}
    \Statex
    \State $s\_params \gets \texttt{sample\_scaler\_params}(\texttt{sampler})$ \label{step:sample_scaler_params}
    \State $s \gets \texttt{instantiate\_scaler}(s\_params)$ \label{step:instantiate_scaler}
    \State $t\_params \gets \texttt{sample\_transformer\_params}(\texttt{sampler})$ \label{step:sample_transformer_params}
    \State $t \gets \texttt{instantiate\_transformer}(t\_params)$ \newline \hspace*{3em} \textbf{or} NONE \label{step:instantiate_transformer}
    \State $dr\_params \gets \texttt{sample\_dim\_reduction\_params}(\texttt{sampler})$ \label{step:sample_dim_reduction_params}
    \State $dr \gets \texttt{instantiate\_dim\_reduction}(dr\_params)$ \newline \hspace*{3em}\textbf{or} NONE \label{step:instantiate_dim_reduction}
    \Statex     
    \State $pipeline \gets [s, t, dr]$ \textbf{or} $[s, NONE, NONE]$ \label{step:construct_pipeline}
    \State $T_{cv}, D_{train}, D_{test} \gets \text{apply data partitioning to } D$ \label{step:data_partitioning}
    \Statex
    \State $D_{train}'$ $\gets$ \texttt{pipeline\_apply}($D_{train}$, \texttt{pipeline}) \label{step:apply_pipeline_train}
    \State $T_{cv}'$ $\gets$ \texttt{pipeline\_apply}($T_{cv}$, \texttt{pipeline}) \label{step:apply_pipeline_cv}
    \Statex
    \State $CV_{metrics} \gets \texttt{cross\_validate}(m', T_{cv}')$ \label{step:cross_validate}
    \State $rmse_{cv} \gets \texttt{average}(CV_{metrics}.\texttt{rmse\_values})$ \label{step:average_rmse_cv}
    \State $std\_dev_{cv} \gets \texttt{average}(CV_{metrics}.\texttt{std\_dev\_values})$ \label{step:average_std_dev_cv}
    \Statex
    \State $m'_{train}$ $\gets$ \texttt{train}($m'$, $D_{train}'$) \label{step:train_model}
    \State $rmsep$, $\sigma_{error, test}$ $\gets$ \texttt{evaluate}($m'_{train}$, $D_{test}$) \label{step:evaluate_model}
    \Statex
    % function that stores the metrics
    \State \texttt{store\_metrics}($t$, $m'$, $pipeline$, $rmse\_cv$, \newline \hspace*{8em}$std\_dev\_cv$, $rmsep$, $\sigma_{error, test}$) \label{step:store_metrics}
\EndFor
\State \Return \texttt{rmse\_cv}, \texttt{std\_dev\_cv}, \texttt{rmsep}, $\sigma_{error, test}$ \label{step:return_metrics}
\end{algorithmic}
\end{algorithm}
% create names for evaluation metrics
% 



% Optuna for its flexibility and efficiency in exploring the vast search space of configurations.
% Optuna allows for great flexiblity in exploring various search spaces due to its modular design.
% In addition to this modularity, Optuna uses a "define-by-run" optimization strategy.
% Rather than being confined to a fixed order and range of exploration, Optuna can dynamically adjust regions based on the results of previous trials.
% Introduction of the optimization framework
    % What is the goal of the optimization framework?
    % Why Optuna?
        % Explanation of what optuna is and why it is useful for our purposes
        % Should elaborate on this:
            % This framework facilitates automated hyperparameter optimization, allowing us to  efficiently explore a vast search space of model and preprocessing configurations.
% how did we do it?
    % Maybe talk about how we designed it in this way
    % pros cons ** maybe
    % Considerations at the very least
    % Talk about how the objective function was constructed, the order of things, considerations with this, the fact that we are trying to minimize and what metrics we are minimizing
    % 
% show diagram or pseudocode
% Explain why we did it this way
    % Why did we choose this approach?
    % What are the benefits of this approach?
    % How does this approach help us achieve our goal?

   
    
% Next, we implemented an experimental framework using the Optuna optimization library~\cite{optuna_2019}.
% This framework facilitates automated hyperparameter optimization, allowing us to efficiently explore a vast search space of model and preprocessing configurations.
% The specifics of this framework are discussed in Section~\ref{sec:optimization_framework}.


% Optuna uses python syntax instead of some propriety syntax
% Optuna brings the optimization into the space of the program rather than outside
    % Meaning you can embed it directly in functions etc.
    % Easier to debug
    % Can use python language such as looping etc.
% Instead of having parameters set you change them so they are sampled using the suggest_* methods
    % This is also helpful because it allows you to optimize any parameter, even for preprocessing etc. in the same objective function
% There are two parts of the hyperparameters optimizer: Sampling strategy and Pruning strategy
    % Sampling strategy determines where to look
        % Uses bayesian filtering to find places where it has had the best results and focus in on those
        % As Optuna tries to minimize/maximize the objective function it focuses in on the best areas and makes more trials there
        % There are multiple samplers - Even the traditional ones
        % Choosing the right sampler is sort of a heuristic 
    % Pruning strategy can terminate trials early that are not promising so that the compute can be dedicated to more promising trials
        % Basically trials that have a slow start and will never be able to make up for that slow start, are pruned
% Optuna is very easy to scale up. It allows you to use a single database across multiple machines
    % This means that Optuna will let you use multiple computers to optimize over the search space at the same time
    % Its called asynchronous parallelization of trials - One trial could start later than the other
    % Near linear scaling with the number of machines
% Has tools to help you determine most important parameters to avoid curse of dimensionality
    % get_param_importances
    % It works by running a small number of trials and then returns this information to you
    % Helps you dial in where optuna should be focusing to get most optimal results

\section{PyHAT Contribution}\label{sec:pyhat_contribution}
As part of our work, we have made several contributions to \gls{pyhat}. 
We describe these contributions here.
\gls{pyhat} offers a user-friendly interface designed for performing machine learning and data analysis tasks specifically for hyperspectral data.
Our collaboration was initiated through a series of discussions with two members from \gls{usgs} that are responsible for \gls{pyhat}, wherein we identified mutual challenges and opportunities for integrating our solutions into the tool.

The largest contribution involved the integration of an automatic outlier detection method into \gls{pyhat}.
This method calculates the Mahalanobis distance for each data point and uses the chi-squared distribution to establish a threshold.
Any datapoint exceeding this threshold is considered an outlier and removed from the dataset.
Utilizing two intermediary \gls{pls} models, one as a reference and the other to evaluate the impact of outlier removal, the method iteratively identifies and eliminates outliers while assessing the performance of the second model against the reference model. 
If the second model demonstrates improved performance compared to the reference model, it replaces the reference model, and the process continues until no further significant improvement is detected. 
To conserve computational resources, the method halts if the error of the second model increases relative to the reference model, thus providing an early stopping mechanism.

This contribution also included the development of a graphical user interface (GUI) component for the existing \gls{pyhat} GUI to configure and visualize the outlier removal process.
This included utilities to select a threshold, select a given oxide for which to perform outlier removal, and a logging mechanism to display the number of outliers removed at each iteration in the GUI.

Another contribution made to \gls{pyhat} involved a fix of an important functionality in their Joint Approximation Diagonalization of Eigen-matrices (JADE) implementation.
The fix provided the ability to properly identify which of the original data points has the highest correlation with each independent component produced by JADE.
The correlation scores produced by this functionality can be used in a regression context, where a linear model learns the coefficients that best fit the relationship between the independent components and the original data points.

Finally, we made some contributions to improve the performance of various processes in \gls{pyhat}.
At the time of writing, all contributions has been demonstrated to work as intended to the two \gls{usgs} members responsible for managing \gls{pyhat} and are undergoing final review.

 
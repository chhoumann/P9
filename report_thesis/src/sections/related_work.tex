\section{Related Work}\label{sec:related-work}
In addressing the challenge of predicting major oxide compositions from \gls{libs} data, our investigation intersects with a broad spectrum of existing research.
Key strategies include the integration of machine learning and deep learning models, the incorporation of domain knowledge, effective preprocessing techniques, and dimensionality reduction methods.
These approaches collectively aim to manage the complexities inherent in \gls{libs} data and improve predictive performance.
We review existing and relevant work through a thematic taxonomy, highlighting their potential applications in our study.

\subsection{Machine Learning Models in LIBS Analysis}
Several studies have applied machine learning models to analyze \gls{libs} data, aiming to predict major oxide compositions with high accuracy.

\citet{andersonPostlandingMajorElement2022} utilized machine learning models to quantify major oxides on Mars using the SuperCam instrument on the Perseverance rover.
Their approach involved extensive preprocessing and normalization of LIBS spectra, followed by the development of multivariate regression models for each oxide.
They demonstrated that blending different models, such as \gls{gbr} and \gls{pls} for \ce{SiO2}, could improve prediction accuracy.
Their study serves as a benchmark for model performance on LIBS spectra and offers insights into model selection for similar datasets.

\citet{yangConvolutionalNeuralNetwork2022} demonstrated the effectiveness of a deep \gls{cnn} for classifying geochemical samples using \gls{libs} spectra collected at varying distances.
Their model outperformed traditional machine learning approaches, emphasizing the potential of \gls{cnn}s for geochemical sample identification in planetary exploration missions like China's Tianwen-1.

\subsection{Hybrid and Domain-Knowledge-Driven Models}
Incorporating domain knowledge into machine learning models can significantly enhance their interpretability and performance.

\citet{song_DF-K-ELM} introduced a hybrid model, \gls{df}-\gls{k-elm}, which integrates domain knowledge-based spectral lines with kernel extreme learning machines.
This method was particularly effective across multiple regression tasks, demonstrating improved accuracy and generalizability.
The integration of domain-specific insights allowed the model to adhere more closely to the physical principles underlying \gls{libs} quantification, making it highly relevant for applications requiring model interpretability.

\citet{sunMachineLearningTransfer2021} applied transfer learning to \gls{libs} spectral data analysis, significantly improving model performance in rock classification on Mars.
By leveraging knowledge from one domain to address related problems in another, their approach addressed the physical matrix effect, enhancing the robustness of the models for rock classification.

\subsection{Preprocessing and Feature Engineering}
Effective preprocessing and feature engineering are critical for enhancing the robustness of \gls{libs} models.

\citet{jeonEffectsFeatureEngineering2024} investigated the effects of various feature-engineering techniques on the robustness of \gls{libs} models for steel classification.
They developed a remote \gls{libs} system to classify six steel types, using various feature-engineering and machine learning algorithms, including \gls{svm} and \gls{fcnn}, to handle different laser energies in test datasets.
They found that using intensity ratios, which involve comparing specific spectral line intensities, significantly improved model robustness under varying measurement conditions.
This approach effectively filtered out noise and enhanced the model's performance, demonstrating the importance of appropriate feature-engineering method.

\citet{Huang2015AnEA} conducted a systematic analysis of data preprocessing techniques, emphasizing the need for tailored strategies to enhance model accuracy.
Evaluating methods such as feature selection, case selection, scaling, and missing data treatments, they demonstrate that the effectiveness of these techniques varies markedly across different datasets and machine learning algorithms.
Their findings underscore the interdependent relationship between preprocessing techniques and model selection, which is crucial for optimizing predictive performance.

\subsection{Dimensionality Reduction Techniques}
Dimensionality reduction techniques such as \gls{pca} play a crucial role in managing the high-dimensional nature of \gls{libs} data.

\citet{rezaei_dimensionality_reduction} explored various machine learning techniques combined with \gls{pca} for dimensionality reduction. Their results demonstrated that \gls{svr} with \gls{pca} performed the best for different elements, highlighting the effectiveness of dimensionality reduction techniques in improving model performance by managing the complexities of high-dimensional data.

The study by \citet{liuComparisonQuantitativeAnalysis2022} explores the use of \gls{marscode} \gls{libs} for quantitative analysis of olivine in a simulated Martian atmosphere, focusing on multivariate analysis methods to address challenges posed by \gls{libs} data, such as high dimensionality and multicollinearity.
The methods evaluated include \gls{ulr}, \gls{mvlr}, \gls{pcr}, \gls{plsr}, ridge regression, \gls{lasso}, \gls{enet}, and \gls{bpnn}.
The findings demonstrate the effectiveness of dimension reduction techniques, especially \gls{plsr}, and nonlinear analysis for improving quantitative analysis accuracy of olivine using \gls{libs} data.
This approach is particularly relevant to our work due to the focus on advanced statistical methods and machine learning algorithms for handling complex, high-dimensional \gls{libs} data, aligning with our objectives of improving accuracy and robustness in predicting major oxide compositions.
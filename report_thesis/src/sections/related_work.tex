\section{Related Work}
\citet{andersonPostlandingMajorElement2022} experimented with different machine learning models for quantifying major oxides on Mars using the SuperCam instrument on the Mars 2020 Perseverance rover.
They discuss preprocessing, normalization of \gls{libs} spectra, and the development of multivariate regression models to predict major element compositions.
For each oxide, they tested different models and selected the best performing one.
In some cases, they used a blend of models to improve the predictions.
The models they tested include: \gls{ols}, \gls{pls}, \gls{lasso}, Ridge, \gls{enet}, \gls{omp}, \gls{svr}, \gls{rf}, \gls{gbr}, and local \gls{enet} and blended submodels.
For \ce{SiO2}, they used a blend of \gls{gbr} and \gls{pls} models.
Interestingly, they found that PLS performed better at longer distances (4.25m), but GBR was better at 3m.
For \ce{TiO2}, they selected the \gls{rf} model for its superior performance at 4.25m and overall lower \gls{rmsep}.
For \ce{Al2O3}, they used an average of predictions from four models (Local \gls{enet}, \gls{rf}, two variants of \gls{pls}) to obtain the lowest \gls{rmsep}.
For \ce{FeO_T}, they initially selected \gls{rf} but later replaced it with \gls{gbr} due to its more realistic stoichiometry predictions for high-\ce{Ca} pyroxenes and overall performance.
For \ce{MgO}, they selected \gls{gbr} for having the lowest \gls{rmsep} and avoiding negative predictions, despite slightly overpredicting \ce{MgO} for high concentration samples.
For \ce{CaO}, they used a blend of \gls{rf} and \gls{pls} to address the bimodal distribution of \ce{CaO} predictions by the \gls{rf} model alone.
For \ce{Na2O}, they used a blend of \gls{gbr} and \gls{lasso} models to utilize \gls{gbr}'s accuracy at low concentrations and \gls{lasso}'s superior predictions at higher concentrations.
For \ce{K2O}, they selected \gls{lasso} for its better performance on high \ce{K2O} samples, despite the averaged model of five algorithms showing slight improvements at lower concentrations.
The findings of this paper are significant to us because they provide a benchmark for the performance of different machine learning models on \gls{libs} spectra.
We can use this information to guide our model selection and to compare our results with theirs.
Additionally, we might want to try out different models from the ones they tested to see if we can improve the predictions further, or perhaps find a model that is more suitable for our specific use case.
Also, SuperCam being the successor to \gls{chemcam} means that the findings of this paper are directly relevant to our work.

\citet{song_DF-K-ELM} present a novel approach to enhance the performance and interpretability of machine learning models in the context of \gls{libs} quantification.
The authors use "knowledge-based spectral lines, related to analyte compositions, to construct a linear physical principle based model and adopts K-ELM to account for the residuals of the linear model."
This method stands out by offering an intuitive explanation of how knowledge-based spectral lines impact prediction results, thereby enhancing model interpretability without compromising model complexity.
DF-K-ELM was tested across 10 regression tasks based on 3 \gls{libs} datasets, comparing its performance against six baseline methods using \gls{rmsep} as the evaluation metric.
They have 3 coal datasets, and they do regression tasks involving carbon, ash, volatile matter, and heat value analysis.
It achieved the best performance in 4 tasks and the second-best in 2 tasks, demonstrating its efficacy.
Incorporation of domain knowledge not only improved the accuracy of the models but also enhanced their generalizability across different tasks.
The method's design allows for a more interpretable machine learning model that adheres closer to the physical principles underlying \gls{libs} quantification.
The integration of domain knowledge into machine learning models addresses two critical challenges: improving the interpretability of complex models and enhancing their performance by leveraging specific domain insights.
The approach demonstrates a practical application of kernel extreme learning machines combined with domain-specific insights.
This is particularly valuable in fields like spectroscopy, where understanding the relationship between the spectral data and the analyte concentration is vital.
The DF-K-ELM method showcases how hybrid models can outperform traditional machine learning approaches.
The approach demonstrates a practical application of kernel extreme learning machines combined with domain-specific insights.
This is very relevant to our work considering interpretability is a key requirement for NASA and something they considered when choosing the PLS model for the \gls{chemcam} instrument.

\citet{rezaei_dimensionality_reduction} explore a variety of statistical and machine learning methods, including \gls{mlr}, \gls{svr}, \gls{ksvr}, and \gls{ann}, alongside their integrations with \gls{pca} to reduce dimensionality and improve model performance.
They use \gls{mse} and \gls{mae} as evaluation metrics to compare the performance of the models.
This paper clearly demonstrates the effectiveness of dimensionality reduction techniques in improving the performance of machine learning models because it compares the performance of many models with and without \gls{pca}: \gls{ann}, \gls{mlr}, \gls{svr}, \gls{ksvr}, \gls{pca}-\gls{ann}, \gls{pca}-\gls{mlr}, \gls{pca}-\gls{svr}, and \gls{pca}-\gls{ksvr}.
For all elements, a variant SVR performs the best.
For \ce{Si}, SVR performs the best.
For \ce{Zn}, PCA-SVR performs the best.
For the rest of the elements, \gls{pca}-\gls{ksvr} performs the best.
The superiority of \gls{ksvr} is attributed to the its ability to handle non-linear relationships in the data effectively, especially when combined with PCA's capability to compress and simplify the input data by focusing on the most relevant variations.

\citet{yang_laser-induced_2022} present a study on the application of a deep \gls{cnn} for classifying geochemical samples using \gls{libs}, with a particular focus on planetary exploration missions such as China's Tianwen-1 Mars mission.
The authors demonstrate the effectiveness of a deep CNN in classifying geochemical standard samples using \gls{libs} spectra collected at varying distances.
This addresses the challenge of spectral differences induced by distance, showcasing that \gls{cnn} can learn to classify samples without the need for traditional spectral preprocessing or distance correction.
Using a dataset of over 18,000 \gls{libs} spectra from 39 geochemical standard samples, the study compares the \gls{cnn} model's performance against four other machine learning models: \gls{bpnn} \gls{svm}, \gls{lda}, and \gls{lr}.
The CNN model exhibits superior classification accuracy, emphasizing its potential for geochemical sample identification/classification in planetary exploration.
The paper includes a detailed comparative analysis, proving the \gls{cnn} model's superior performance.
With classification accuracies on the validation set for all models exceeding 95\%, the \gls{cnn} model demonstrated the highest overall accuracy.
This was particularly evident as the training set size increased, indicating the model's robustness to varying distances without requiring distance correction.
Statistical analysis further confirmed the \gls{cnn} model's superiority, with higher average Ncorr values compared to other models.
The CNN model's ability to accurately classify geochemical samples without preprocessing for distance correction is quite impressive.
This is particularly relevant to our work because we are also working with \gls{libs} spectra collected at varying distances.
The comparative analysis underscores the CNN model as a best-fit approach for \gls{libs} data analysis, potentially setting a new standard for future research and applications in the field.
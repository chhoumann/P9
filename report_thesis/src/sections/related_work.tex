\section{Related Work}
\citet{andersonPostlandingMajorElement2022} experimented with different machine learning models for quantifying major oxides on Mars using the SuperCam instrument on the Mars 2020 Perseverance rover.
They discuss preprocessing, normalization of LIBS spectra, and the development of multivariate regression models to predict major element compositions.
For each oxide, they tested different models and selected the best performing one.
In some cases, they used a blend of models to improve the predictions.
The models they tested include: \gls{ols}, \gls{pls}, \gls{lasso}, Ridge, \gls{enet}, \gls{omp}, \gls{svr}, \gls{rf}, \gls{gbr}, and local \gls{enet} and blended submodels.
For \ce{SiO2}, they used a blend of \gls{gbr} and \gls{pls} models.
Interestingly, they found that PLS performed better at longer distances (4.25m), but GBR was better at 3m.
For \ce{TiO2}, they selected the \gls{rf} model for its superior performance at 4.25m and overall lower RMSEP.
For \ce{Al2O3}, they used an average of predictions from four models (Local Elastic Net, \gls{rf}, two variants of \gls{pls}) to obtain the lowest RMSEP.
For \ce{FeO_T}, they initially selected \gls{rf} but later replaced it with \gls{gbr} due to its more realistic stoichiometry predictions for high-Ca pyroxenes and overall performance.
For \ce{MgO}, they selected \gls{gbr} for having the lowest RMSEP and avoiding negative predictions, despite slightly overpredicting \ce{MgO} for high concentration samples.
For \ce{CaO}, they used a blend of \gls{rf} and \gls{pls} to address the bimodal distribution of \ce{CaO} predictions by the \gls{rf} model alone.
For \ce{Na2O}, they used a blend of \gls{gbr} and \gls{lasso} models to utilize \gls{gbr}'s accuracy at low concentrations and \gls{lasso}'s superior predictions at higher concentrations.
For \ce{K2O}, they selected \gls{lasso} for its better performance on high \ce{K2O} samples, despite the averaged model of five algorithms showing slight improvements at lower concentrations.
The findings of this paper are significant to us because they provide a benchmark for the performance of different machine learning models on LIBS spectra.
We can use this information to guide our model selection and to compare our results with theirs.
Additionally, we might want to try out different models from the ones they tested to see if we can improve the predictions further, or perhaps find a model that is more suitable for our specific use case.
Also, SuperCam being the successor to ChemCam means that the findings of this paper are directly relevant to our work.
\subsection{Initial Results}
As described in Section~\ref{sec:proposed_approach}, we conducted a series of initial experiments to evaluate the performance of various machine learning models on the prediction of major oxide compositions from our \gls{libs} dataset.
These experiments aimed to provide a preliminary assessment of the models' performance.
All models were trained on the same preprocessed data using the Norm 3 preprocessing method described in Section~\ref{sec:norm3}.
This ensured that the models' performance could be evaluated under consistent and comparable conditions.

Table~\ref{tab:init_results} presents the results of these experiments, including the \gls{rmsep}, \gls{rmsecv}, standard deviation, standard deviation of the cross-validation, and the mean of these metrics for each model and oxide.
The primary metric of interest is the \gls{rmsecv}, as it provides a measure of the model's predictive accuracy during cross-validation, reflecting its generalization performance on unseen data.
Additionally, the standard deviation of the cross-validation offers insight into the variability of the model's performance across different folds, and the \gls{rmsep} indicates the model's predictive accuracy on a separate test set.

Ridge regression and \gls{svr} generally exhibit strong performance across different oxides, as evidenced by their relatively low \gls{rmsecv} values.
In contrast, \gls{enet}, \gls{ann}, and \gls{cnn} tend to perform the worst, with higher \gls{rmsecv} values.
Models such as \gls{rf} and \gls{gbr} exhibit moderate performance, often surpassing simpler linear models but not as consistently as \gls{svr}.

For $\ce{SiO2}$, \gls{svr} demonstrates the lowest \gls{rmsecv} of 4.908.
Ridge regression and \gls{rf} also perform well, with \gls{rmsecv} values of 5.005 and 5.304, respectively.
The standard deviations for these models are relatively low, indicating stable performance.

In the case of $\ce{TiO2}$, \gls{etr} exhibits the lowest \gls{rmsecv} of 0.427, signifying exceptional performance, with \gls{svr} and \gls{ngboost} also showing strong predictive capabilities, with \gls{rmsecv} values of 0.463 and 0.433, respectively.
he low standard deviation of the cross-validation for these models indicates consistent performance across folds.

For $\ce{Al2O3}$, \gls{ngboost} with an \gls{rmsecv} of 2.291 and ridge regression with an \gls{rmsecv} of 3.115 achieve relatively low \gls{rmsecv}, highlighting their effectiveness in predicting this oxide, with \gls{svr} also performing well with an \gls{rmsecv} of 2.700.
he standard deviation analysis reveals that \gls{ngboost} has the least variability, making it a more reliable model for this oxide.

\gls{svr} emerges as the best-performing model for $\ce{FeO_T}$ with the lowest \gls{rmsecv} of 2.847, while ridge regression with an \gls{rmsecv} of 3.385 and \gls{ngboost} with an \gls{rmsecv} of 3.561 also demonstrate competent performance.
The standard deviation values indicate that while \gls{svr} has lower variability, the other models also maintain reasonable consistency.

For $\ce{MgO}$, \gls{svr} exhibits the best performance with the lowest \gls{rmsecv} of 1.426, and ridge regression with an \gls{rmsecv} of 1.509 and \gls{ngboost} with an \gls{rmsecv} of 1.578 are also effective.
The standard deviation metrics suggest that these models are more stable, with \gls{svr} showing the least variation.

\gls{gbr} shows the lowest \gls{rmsecv} of 1.468 for $\ce{CaO}$, indicating strong performance, with ridge regression having an \gls{rmsecv} of 1.513 and \gls{etr} having an \gls{rmsecv} of 1.503 also performing well.
Standard deviation values also highlight \gls{gbr} and \gls{etr} as having consistent performance.

\gls{etr} demonstrates the best performance for $\ce{Na2O}$ with the lowest \gls{rmsecv} of 1.028, while ridge regression with an \gls{rmsecv} of 1.221 and \gls{svr} with an \gls{rmsecv} of 1.096 also perform reasonably well.
The standard deviation of the cross-validation for \gls{etr} is particularly low, underscoring its robustness.

For $\ce{K2O}$, \gls{etr} again shows strong performance with the lowest \gls{rmsecv} of 0.681, with \gls{svr} having an \gls{rmsecv} of 0.690 and ridge regression having an \gls{rmsecv} of 0.668 also performing well.
The standard deviation analysis supports these findings, indicating reliable performance for these models.

To summarize, \gls{svr} stands out as the best overall model based on its average performance, with a mean \gls{rmsecv} of 1.958. This is closely followed by \gls{ngboost} with a mean \gls{rmsecv} of 2.017 and ridge regression with a mean \gls{rmsecv} of 2.111.
When examining the \gls{rmsep} values, \gls{svr} also demonstrates strong predictive accuracy on independent test sets, reinforcing its effectiveness.
Additionally, the standard deviation of the cross-validation for \gls{svr} is generally low, indicating stable and consistent performance across different subsets of data.

\gls{ngboost} and ridge regression also show robust performance, with relatively low standard deviations.
Specifically, \gls{ngboost} exhibits consistent performance with low variability, making it a dependable model for various oxides.
Ridge regression, while slightly higher in \gls{rmsecv}, still maintains good performance with low standard deviation values.

Overall, \gls{svr}, ridge regression, and \gls{ngboost} are consistently among the top-performing models for multiple oxides, as evidenced by their low \gls{rmsecv}, \gls{rmsep}, and standard deviation values.

\begin{table*}[]
\centering
\caption{Initial results for the different models and metrics.}
\resizebox{1\textwidth}{!}{%
\begin{tabular}{l|cccc|cccc|cccc}
Model & \multicolumn{4}{c}{Ridge} & \multicolumn{4}{c}{\gls{lasso}} & \multicolumn{4}{c}{\gls{enet}} \\
Metric & \multicolumn{1}{c}{RMSEP} & \multicolumn{1}{c}{RMSECV} & \multicolumn{1}{c}{Std. dev.} & \multicolumn{1}{c}{Std. dev. CV} & \multicolumn{1}{c}{RMSEP} & \multicolumn{1}{c}{RMSECV} & \multicolumn{1}{c}{Std. dev.} & \multicolumn{1}{c}{Std. dev. CV} & \multicolumn{1}{c}{RMSEP} & \multicolumn{1}{c}{RMSECV} & \multicolumn{1}{c}{Std. dev.} & \multicolumn{1}{c}{Std. dev. CV} \\
\hline
$\ce{SiO2}$ & 4.104 & 5.004 & 4.108 & 5.005 & 4.412 & 5.431 & 4.417 & 5.437 & 4.412 & 5.431 & 4.417 & 5.437 \\
$\ce{TiO2}$ & 0.424 & 0.470 & 0.413 & 0.469 & 0.398 & 0.556 & 0.389 & 0.555 & 0.398 & 0.556 & 0.389 & 0.555 \\
$\ce{Al2O3}$ & 2.322 & 2.913 & 2.324 & 2.888 & 2.349 & 3.063 & 2.352 & 3.044 & 2.349 & 3.063 & 2.352 & 3.044 \\
$\ce{FeOT}$ & 2.068 & 3.173 & 2.070 & 3.122 & 2.236 & 3.490 & 2.238 & 3.440 & 2.236 & 3.490 & 2.238 & 3.440 \\
$\ce{MgO}$ & 1.150 & 1.509 & 1.152 & 1.492 & 1.267 & 1.682 & 1.249 & 1.661 & 1.267 & 1.682 & 1.249 & 1.661 \\
$\ce{CaO}$ & 1.844 & 1.485 & 1.833 & 1.478 & 1.963 & 1.554 & 1.962 & 1.549 & 1.963 & 1.554 & 1.962 & 1.549 \\
$\ce{Na2O}$ & 0.632 & 1.089 & 0.633 & 1.084 & 0.625 & 1.114 & 0.616 & 1.111 & 0.588 & 1.085 & 0.587 & 1.082 \\
$\ce{K2O}$ & 0.651 & 0.668 & 0.645 & 0.668 & 0.638 & 0.859 & 0.629 & 0.856 & 0.638 & 0.859 & 0.629 & 0.856 \\
\hline
Mean & 1.649 & 2.039 & 1.647 & 2.026 & 1.736 & 2.219 & 1.732 & 2.207 & 1.731 & 2.215 & 1.728 & 2.203 \\
\hline
Model & \multicolumn{4}{c}{\gls{pls}} & \multicolumn{4}{c}{\gls{svr}} & \multicolumn{4}{c}{\gls{rf}} \\
Metric & \multicolumn{1}{c}{RMSEP} & \multicolumn{1}{c}{RMSECV} & \multicolumn{1}{c}{Std. dev.} & \multicolumn{1}{c}{Std. dev. CV} & \multicolumn{1}{c}{RMSEP} & \multicolumn{1}{c}{RMSECV} & \multicolumn{1}{c}{Std. dev.} & \multicolumn{1}{c}{Std. dev. CV} & \multicolumn{1}{c}{RMSEP} & \multicolumn{1}{c}{RMSECV} & \multicolumn{1}{c}{Std. dev.} & \multicolumn{1}{c}{Std. dev. CV} \\
\hline
$\ce{SiO2}$ & 4.141 & 5.701 & 4.145 & 5.693 & 3.552 & 4.908 & 3.555 & 4.908 & 3.715 & 5.304 & 3.699 & 5.292 \\
$\ce{TiO2}$ & 0.452 & 0.531 & 0.441 & 0.530 & 0.461 & 0.463 & 0.455 & 0.462 & 0.331 & 0.427 & 0.321 & 0.425 \\
$\ce{Al2O3}$ & 2.073 & 3.322 & 2.061 & 3.302 & 1.931 & 2.700 & 1.934 & 2.693 & 2.076 & 2.443 & 2.079 & 2.433 \\
$\ce{FeOT}$ & 3.222 & 3.117 & 3.221 & 3.114 & 1.823 & 2.847 & 1.814 & 2.809 & 2.091 & 3.091 & 2.073 & 3.053 \\
$\ce{MgO}$ & 1.106 & 1.296 & 1.103 & 1.296 & 0.789 & 1.426 & 0.785 & 1.419 & 0.911 & 1.742 & 0.904 & 1.731 \\
$\ce{CaO}$ & 1.937 & 1.813 & 1.923 & 1.792 & 1.626 & 1.532 & 1.594 & 1.508 & 1.765 & 1.503 & 1.754 & 1.499 \\
$\ce{Na2O}$ & 0.545 & 0.908 & 0.536 & 0.906 & 0.742 & 1.096 & 0.725 & 1.086 & 0.420 & 1.028 & 0.421 & 1.023 \\
$\ce{K2O}$ & 0.774 & 0.650 & 0.772 & 0.646 & 0.567 & 0.690 & 0.555 & 0.689 & 0.524 & 0.681 & 0.476 & 0.676 \\
\hline
Mean & 1.781 & 2.167 & 1.775 & 2.160 & 1.436 & 1.958 & 1.427 & 1.947 & 1.479 & 2.027 & 1.466 & 2.017 \\
\hline
Model & \multicolumn{4}{c}{\gls{ngboost}} & \multicolumn{4}{c}{\gls{gbr}} & \multicolumn{4}{c}{\gls{xgboost}} \\
Metric & \multicolumn{1}{c}{RMSEP} & \multicolumn{1}{c}{RMSECV} & \multicolumn{1}{c}{Std. dev.} & \multicolumn{1}{c}{Std. dev. CV} & \multicolumn{1}{c}{RMSEP} & \multicolumn{1}{c}{RMSECV} & \multicolumn{1}{c}{Std. dev.} & \multicolumn{1}{c}{Std. dev. CV} & \multicolumn{1}{c}{RMSEP} & \multicolumn{1}{c}{RMSECV} & \multicolumn{1}{c}{Std. dev.} & \multicolumn{1}{c}{Std. dev. CV} \\
\hline
$\ce{SiO2}$ & 4.112 & 5.071 & 4.081 & 5.010 & 3.576 & 4.995 & 3.479 & 4.922 & 3.953 & 4.898 & 3.926 & 4.876 \\
$\ce{TiO2}$ & 0.340 & 0.433 & 0.333 & 0.430 & 0.474 & 0.449 & 0.473 & 0.446 & 0.334 & 0.437 & 0.328 & 0.436 \\
$\ce{Al2O3}$ & 1.931 & 2.291 & 1.933 & 2.282 & 1.894 & 2.518 & 1.891 & 2.511 & 1.912 & 2.198 & 1.913 & 2.193 \\
$\ce{FeOT}$ & 1.588 & 3.561 & 1.590 & 3.530 & 1.594 & 3.069 & 1.596 & 3.068 & 1.848 & 3.020 & 1.838 & 3.002 \\
$\ce{MgO}$ & 0.849 & 1.578 & 0.845 & 1.574 & 0.964 & 1.766 & 0.960 & 1.763 & 0.905 & 1.781 & 0.901 & 1.771 \\
$\ce{CaO}$ & 1.740 & 1.610 & 1.723 & 1.602 & 1.768 & 1.468 & 1.769 & 1.468 & 1.765 & 1.467 & 1.749 & 1.457 \\
$\ce{Na2O}$ & 0.416 & 0.921 & 0.415 & 0.916 & 0.481 & 1.130 & 0.481 & 1.123 & 0.387 & 1.071 & 0.387 & 1.062 \\
$\ce{K2O}$ & 0.582 & 0.675 & 0.545 & 0.673 & 0.727 & 0.609 & 0.719 & 0.610 & 0.547 & 0.658 & 0.511 & 0.657 \\
\hline
Mean & 1.445 & 2.017 & 1.433 & 2.002 & 1.435 & 2.001 & 1.421 & 1.989 & 1.456 & 1.941 & 1.444 & 1.932 \\
\hline
Model & \multicolumn{4}{c}{\gls{etr}} & \multicolumn{4}{c}{\gls{ann}} & \multicolumn{4}{c}{\gls{cnn}} \\
Metric & \multicolumn{1}{c}{RMSEP} & \multicolumn{1}{c}{RMSECV} & \multicolumn{1}{c}{Std. dev.} & \multicolumn{1}{c}{Std. dev. CV} & \multicolumn{1}{c}{RMSEP} & \multicolumn{1}{c}{RMSECV} & \multicolumn{1}{c}{Std. dev.} & \multicolumn{1}{c}{Std. dev. CV} & \multicolumn{1}{c}{RMSEP} & \multicolumn{1}{c}{RMSECV} & \multicolumn{1}{c}{Std. dev.} & \multicolumn{1}{c}{Std. dev. CV} \\
\hline
$\ce{SiO2}$ & 3.995 & 5.230 & 3.970 & 5.225 & 4.664 & 7.025 & 4.670 & 6.981 & 4.662 & 6.061 & 4.626 & 6.046 \\
$\ce{TiO2}$ & 0.330 & 0.439 & 0.321 & 0.438 & 0.436 & 0.543 & 0.431 & 0.540 & 0.571 & 0.634 & 0.565 & 0.628 \\
$\ce{Al2O3}$ & 1.845 & 2.368 & 1.847 & 2.359 & 2.624 & 3.049 & 2.628 & 3.026 & 2.482 & 2.871 & 2.457 & 2.854 \\
$\ce{FeOT}$ & 2.144 & 3.299 & 2.126 & 3.257 & 2.534 & 3.836 & 2.497 & 3.748 & 2.588 & 4.584 & 2.521 & 4.488 \\
$\ce{MgO}$ & 0.906 & 1.755 & 0.895 & 1.738 & 1.315 & 1.818 & 1.300 & 1.768 & 1.292 & 2.892 & 1.280 & 2.857 \\
$\ce{CaO}$ & 1.837 & 1.515 & 1.831 & 1.510 & 1.799 & 1.633 & 1.772 & 1.634 & 2.009 & 2.142 & 2.008 & 2.099 \\
$\ce{Na2O}$ & 0.411 & 1.031 & 0.409 & 1.028 & 0.539 & 1.095 & 0.532 & 1.091 & 0.656 & 1.364 & 0.657 & 1.357 \\
$\ce{K2O}$ & 0.591 & 0.642 & 0.540 & 0.636 & 0.659 & 0.850 & 0.640 & 0.845 & 0.783 & 1.684 & 0.742 & 1.657 \\
\hline
Mean & 1.507 & 2.035 & 1.492 & 2.024 & 1.821 & 2.481 & 1.809 & 2.454 & 1.880 & 2.779 & 1.857 & 2.748 \\
\hline
\end{tabular}%
}
\label{tab:init_results}
\end{table*}
